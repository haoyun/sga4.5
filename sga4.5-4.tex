% !TEX root = sga4.5.tex

\chapter{La classe de cohomologie associée à un cycle}\label{IV}




Cet exposé est inspiré de notes de Grothendieck, qui formaient un état 0 
de \cite[IV]{sga5}. On y définit la classe de cohomologie d'un cycle $X$ dans 
un schéma séparé lisse de type fini sur un corps et on prouve que 
l'intersection correspond au cup-produit.

Le Chapitre \ref{IV:1} contient quelques sorites généraux. Au Chapitre \ref{IV:2}, on 
définit la classe d'un cycle, dans plusieurs situations plus générales 
que celle dite plus haut. La principale compatibilité non considérée est 
celle entre image directe d'un cycle et morphisme trace en cohomologie. Au 
Chapitre \ref{IV:3}, on déduit de ce formalisme la formule des traces de Lefschetz pour 
un endomorphisme à points fixes isolés d'un schéma propre et lisse sur $k$ 
algébriquement clos -- et pour un endomorphisme de Frobenius d'une courbe. 

Nous faisons les conditions suivantes:
\begin{enumerate}[\indent 1)]
  \item ``schéma'' signifie schéma noethérien séparé (ceci est 
    largement un hypothèse de commodité).
  \item Dans le Chapitre \ref{IV:2}, on fixe un entier $n$, et $n$ est inversible sur 
    tous les schémas considérés.
  \item Dans le Chapitre \ref{IV:3}, on fixe un nombre premier $\ell$, et $\ell$ est 
    inversible sur tout les schémas considérés. La cohomologie utilisée 
    est toujours la cohomologie $\ell$-adique:
    \[
      \h^\bullet(X) = \dQ_\ell\otimes_{\dZ_\ell} \varprojlim \h^\bullet(X,\dZ/\ell^n) \text{.}
    \]
\end{enumerate}
Classes de cohomologie de cycles, morphismes traces, \ldots sont définis par 
passage à limite à partir du cas de coefficients finis $\dZ/\ell^n$ (cf. 
\cite[VI]{sga5}).




















\section{Cohomologie à support et cup--produits}\label{IV:1}

Ce paragraphe contient des rappels de topologie générale, que le 
lecteur est invité à ne consulter qu'au fur et à mesure des besoins.










\subsection{\texorpdfstring{$H^1$}{H1} et torseurs}\label{IV:1-1}





\subsubsection{}\label{IV:1-1-1}

Soit $\sF$ un faisceau abélien sur un site $X$. On sait que $\h^1(X,\sF)$ 
classifie les $\sF$-torseurs sur $X$. Nous normaliserons ($=$choisirons le 
signe) de l'isomorphisme $($ensemble des classes d'isomorphisme de 
$F$-torseurs$)\to \h^1(X,\sF)$ de telle sorte que pour toute suite exacte 
$0\to \sF\xrightarrow\alpha\sG\xrightarrow\beta\sH\to 0$ et tout 
$h\in\h^0(X,\sH)$, le $\sF$-torseur $\beta^{-1}(h)\subset \sG$, sur lequel 
$F$ agit par $(f,x)\mapsto \alpha(f)+x$, soit de classe $\partial h$. 





\subsubsection{}\label{IV:1-1-2}

Soit $\sP$ un $\sF$-torseur. Si $(U_i)$ est un recouvrement ouvert de $X$, et 
$p_i$ un section de $\sP$ sur $U_i$, on associe à $\sP$ le cocycle de 
\v{C}ech
\[
  p_{i j} = p_j - p_i \qquad \text{($p_{i j}\in \h^0(U_i\times U_j,\sF)$).}
\]
Si, selon la règle usuelle, on définit l'application 
$\operatorname{\check H}^\bullet(X,\sF)\to \h^\bullet(X,\sF)$ de telle sorte 
que ce soit un morphisme de $\delta$-foncteurs, l'image de 
$(p_{i j})\in \operatorname{\check H}^1(X,\sF)$ dans $\h^1(X,\sF)$ est la 
classe de $\sP$, telle que normalisée par \ref{IV:1-1-1}. 





\subsubsection{}\label{IV:1-1-3}

La définition de $\h^\bullet(X,\sF)$ est la suivante: pour $\sF^\bullet$ une 
résolution à composantes acycliques de $\sF$, 
$\h^i(X,\sF) = \h^i\Gamma(X,\sF^\bullet)$. La structure de $\delta$-foncteur 
s'obtient en associant à une suite exacte courte de faisceaux une suite 
exacte courte de résolutions qui reste exacte après application du 
foncteur $\Gamma$. Si $\sF^\bullet$ est une résolution de $\sF$, 
l'homomorphisme de connection $\partial$ associé à 
\[\xymatrix{
  0 \ar[r]
    & \sF \ar[r]
    & \sF^0 \ar[r]^-d 
    & \ker(d) \ar[r] 
    & 0
}\]
induit l'opposé de l'isomorphisme de définition 
$\h^1(X,\sF) = \Gamma\left(X,\ker(d)\right)/d \Gamma(X,\sF^0)$. 





\subsubsection{}\label{IV:1-1-4}

Soit $U$ une partie ouverte de $X$ (un sous-faisceau du faisceau final) et 
soit $D$ le ``fermé complémentaire.'' On sait que $\h_D^1(X,\sF)$ 
classifie les $\sF$-torseurs sur $X$, trivialisés sur $U$. Pour toute suite 
exacte courte $0\to \sF\xrightarrow\alpha \sG\xrightarrow\beta\sH\to 0$, et 
toute section à support dans $D$ $h\in\h_D^0(X,\sH)$, le torseur 
$\beta^{-1}(h)$, trivialisé sur $U$ par la section $0$, a pour classe 
$\partial h$.

La suite exacte longue de cohomologie à support
\[\xymatrix{
  \cdots \ar[r]^-\partial 
    & \h_D^i(X,\sF) \ar[r] 
    & \h^i(X,\sF) \ar[r]
    & \h^i(U,\sF) \ar[r]^-\partial 
    & \cdots
}\]
est définie à partir de la suite de foncteurs 
\[\xymatrix{
  0 \ar[r] 
    & \Gamma_D \ar[r] 
    & \Gamma \ar[r] 
    & \Gamma(U,-) \ar[r] 
    & 0
}\]
(exacte sur les faisceaux injectifs). Pour toute section $f\in \h^0(U,\sF)$, 
$\partial f\in \h_D^1(X,\sF)$ est la classe du torseur trivial $\sF$, 
trivialisé sur $U$ par la section $f$. 





\subsubsection{}\label{IV:1-1-5}

Soit $j:U\hookrightarrow X$. Si $\sF$ s'injecte dans $j_* j^* \sF$, la suite 
exacte 
\[\xymatrix{
  0 \ar[r] 
  & \sF \ar[r]
  & j_* j^* \sF \ar[r]
  & j_* j^* \sF / \sF \ar[r]
  & 0
}\]
fournit 
$\partial:\h^0(j_* j^*\sF/\sF) = \h_D^0(j_* j^* \sF/\sF) \to \h_D^1(X,\sF)$. 
L'application composée 
\[\xymatrix{
  \h^0(U,\sF) = \h^0(X,j_* j^*\sF) \ar[r] 
    & \h^0(X,j_* j^*\sF / \sF) \ar[r]^-\partial 
    & \h_D^1(X,\sF)
}\]
est l'opposée de l'application considérée en \ref{IV:1-1-4}. 





\subsubsection{}\label{IV:1-1-6}

On rappelle que si $\sL$ est un faisceau inversible sur un schéma $X$, le 
$\dG_m$-torseur correspondant est le faisceau $\operatorname{Isom}(\sO,\sL)$, 
sur lequel $\dG_m$ agit par 
$(\lambda,f)\mapsto f\circ (\lambda\cdot) = \lambda f$. On rappelle aussi que 
si $D$ est un diviseur de Cartier sur $X$, et $j:U\hookrightarrow X$ 
l'inclusion d'ouvert complémentaire, le faisceau inversible $\sO(D)$ est le 
sous-faisceau de $j_*\sO_U$ formé des sections locales $s$ telles que $s f$ 
soit dans $\sO_X$, pour $f$ une équation locale de $D$.










\subsection{Cup-produits}\label{IV:1-2}

Dans ce numéro, nous développons quelques remarques sur les cup-produits en 
cohomologie à support qui nous reservirons, dans [\nameref{V}], pour relier 
dualité de Poincaré des courbes et autodualité de la jacobienne.

\subsubsection{}\label{IV:1-2-1}

Soient $X$ un site, $Y$ un partie fermée de $X$, et $\sF,\sG$ deux faisceaux 
abéliens sur $X$. Par exemple: $X$ un schéma, $Y$ un sous-schéma fermé 
et $\sF,\sG$ des faisceaux sur $\et X$. Définissons un produit 
\begin{equation*}\tag{1.2.1.1}\label{IV:eq:1-2-1-1}
  \Gamma_Y(X,\sF)\otimes \Gamma(Y,\sG) \to \Gamma_Y(X,\sF\otimes\sG) \text{.}
\end{equation*}
Le produit d'une section $s$, à support dans $Y$, de $\sF$ par une section 
$t$ de $\sG$ sur $Y$ s'obtient comme suit: localement sur $X$, $t$ est la 
restriction à $Y$ d'une section $t'$ de $\sG$ sur $X$, et on forme le produit 
$s\otimes t'$. Il est à support dans $Y$, et ne dépend pas du choix de 
$t'$, ce qui légitime et permet de globaliser la définition.

Soit $i:Y\hookrightarrow X$ le morphisme d'inclusion. L'analogue local de 
\eqref{IV:eq:1-2-1-1} est le produit.
\begin{equation*}\tag{1.2.1.2}\label{IV:eq:1-2-1-2}
  i^!\sF\otimes i^*\sG \to i^!(\sF\otimes\sG) \text{,}
\end{equation*}
dont \eqref{IV:eq:1-2-1-1} se déduit par application de $\Gamma(Y,-)$. 





\subsubsection{}\label{IV:1-2-2}

Dérivons ces flèches. Soient $\sK$, $\sL$ et $\sM$ dans la catégorie 
dérivée, et une application bilinéaire $\sK\lotimes\sL\to\sM$. Par 
dérivation de \eqref{IV:eq:1-2-1-1}, on en déduit 
\begin{equation*}\tag{1.2.2.1}\label{IV:eq:1-2-2-1}
  \eR\Gamma_Y(X,\sK) \lotimes \eR\Gamma(Y,\sL) \to \eR\Gamma_Y(X,\sM) 
\end{equation*}
induisant 
\begin{equation*}\tag{1.2.2.2}\label{IV:eq:1-2-2-2}
  \h_Y^i(X,\sK)\otimes \h^j(Y,\sL) \to \h_Y^{i+j}(X,\sM)
\end{equation*}
(le cup-produit). La flèche locale \eqref{IV:eq:1-2-1-2} fournit 
\begin{equation*}\tag{1.2.2.3}\label{IV:eq:1-2-2-3}
  \eR i^! \sK\lotimes i^*\sL \to \eR i^!\sM \text{,}
\end{equation*}
dont \eqref{IV:eq:1-2-2-1} se déduit par application de $\eR\Gamma(Y,-)$. 





\subsubsection{}\label{IV:1-2-3}

Ci-dessus, j'ai passé sous silence les conflits qu'entraine l'usage dans une 
même formule de dérivations à droite ($\eR\Gamma$) et à gauche 
($\lotimes$). 
\begin{enumerate}[\indent a)]
  \item Au numéro suivant, les dérivés droits considérés seront tous 
    de dimension cohomologique finie. Ceci permet de travailler 
    systématiquement dans les catégories dérivées $\eD^-$. Pour disposer 
    de complexes à la fois plats et flasques, on utilise les résolutions 
    flasques canoniques comme en \cite[XVII]{sga4}. 
  \item Pour une théorie plus générale, il cesse d'être d'interpréter 
    les applications bilinéaires de $\sK$ et $\sL$ dans $\sM$ comme des 
    morphismes de $\sK\lotimes\sL$ dans $\sM$. Par exemple, 
    $\eR\Gamma_Y(X,\sK)\lotimes\eR\Gamma(Y,\sL)$ n'est pas défini si des deux 
    facteurs sont dans $\eD^+$. Une solution est de travailler dans $\eD^+$, de 
    définir 
    \[
      \operatorname{Bil}(\sK,\sL;\sM) = \varinjlim \hom(\sK'\otimes\sL',\sM') \text{,}
    \]
    où la limite est prise sur les quasi-isomorphismes $\sK'\iso \sK$, 
    $\sL'\iso \sL$, $\sM\iso \sM'$ ($\sK'$, $\sL'$, $\sM'$ bornés 
    inférieurement) et où $\hom$ est pour ``morphisme de complexes, à 
    homotopie près,'' et d'utiliser systématiquement de telles applications 
    bilinéaires, sans jamais mentionner de $\otimes$.  
\end{enumerate}




\subsubsection{Deuxième thème}\label{IV:1-2-4}

Soit $U$ un ouvert de $X$ et $j:U\hookrightarrow X$ le morphisme d'inclusion. 
Pour $\sK$ dans la catégorie dérivée, sur $U$, on pose 
$\eR\Gamma_!(U,\sK)=\eR\Gamma(X,j_!\sK)$. Pour $\sK$, $\sL$ et $\sM$ sur $U$, et 
une application bilinéaire $\sK\lotimes\sL\to\sM$, on veut définir 
\begin{equation*}\tag{1.2.4.1}\label{IV:eq:1-2-4-1}
  \eR\Gamma(U,\sK)\lotimes\eR\Gamma_!(U,\sL)\to\eR\Gamma_!(U,\sM) \text{.}
\end{equation*}
Au niveau des faisceaux, et de leurs sections globales, un tel produit se 
déduit de l'isomorphisme 
$j_*\sF\otimes j_!\sG\xleftarrow\sim j_!(\sF\otimes \sG)$, mais il faut prendre 
garde au fait que $\eR\Gamma_!$ n'est en général pas le dérivé du 
foncteur $\Gamma(X,j_!-)$. 

On commence par définir 
\[\xymatrix{
  \eR j_* \sK\lotimes j_! \sL 
    & \ar[l]_-\sim j_!(\sK\lotimes \sL) \ar[r] 
    & j_!\sM \text{.}
}\]
Appliquant $\eR\Gamma(X,-)$, on trouve 
\[\xymatrix{
  \eR\Gamma(U,\sK)\lotimes\eR\Gamma_!(U,\sL) = \eR\Gamma(X,\eR j_*\sK)\lotimes \eR\Gamma(X,j_!\sL) \ar[r] 
    & \eR\Gamma(X,j_! \sM) \text{.}
}\]

Ici encore, le conflit entre la gauche et la droite se résoudra au numéro 
suivant en travaillant dans $\eD^-$. 





\subsubsection{Coda}\label{IV:1-2-5}

Soient $j:U\hookrightarrow X$ une partie ouverte et $i:Y\hookrightarrow U$ une 
partie fermée de $U$. Soit $\bar Y$ une fermé de $X$ tel que 
$\bar Y\cap U=Y$, (par exemple, le complément de $U\setminus Y$). Soit $\sK$ 
sur $U$. On pose $\eR\Gamma_{Y!}(U,\sK)=\eR\Gamma_!(Y,\eR i^!\sK)$ où 
$\eR\Gamma_!$ est relatif à l'inclusion de $Y$ dans $\bar Y$. Pour tout ouvert 
$V$ de $U$, contenant $Y$, $\eR\Gamma_{Y!}(U,\sK)$ s'envoie dans 
$\eR\Gamma_!(V,\sK)$: si on note encore $i$ l'inclusion de $\bar Y$ dans $X$, 
$j$ celle de $Y$ dans $\bar Y$ et $k$ celle de $V$ dans $X$, on a 
$\eR i^!\sK = j^*\eR i^! k_!(k^*\sK)$, d'où un morphisme 
$j_!\eR i^!\sK \to \eR i^! k_!(k^*\sK)$. Lui appliquant 
$\eR\Gamma(\bar Y,-)$, on trouve 
$\eR\Gamma_{Y!}(U,\sK)\to \eR\Gamma_{\bar Y}(k_! k^* \sK) \to \eR\Gamma(k_! k^*\sK) = \eR\Gamma_!(V,\sK)$. 

Pour $\sK,\sL,\sM$ sur $U$, et une application bilinéaire 
$\sK\lotimes\sL\to\sM$, on veut définir 
\begin{equation*}\tag{1.2.5.1}\label{IV:eq:1-2-5-1}
  \h^n(U,\sK)\otimes \h_!^m(Y,\sL) \to \h_{Y!}^{n+m}(U,\sM)
\end{equation*}
(ce dernier groupe lui-même s'envoyant dans $\h_!^{n+m}(V,\sM)$). Dans la 
catégorie dérivée, il s'agit de définir 
\begin{equation*}\tag{1.2.5.2}\label{IV:eq:1-2-5-2}
  \eR\Gamma_Y(U,\sK) \lotimes \eR\Gamma_!(R,\sL) \to \eR\Gamma_{Y!}(U,\sM) \text{.}
\end{equation*}
On identifie $\eR\Gamma_Y$ à $\eR\Gamma(Y,\eR i^!\sK)$. Le produit cherché 
est alors du type \eqref{IV:eq:1-2-4-1} relatif au produit locale 
\eqref{IV:eq:1-2-2-3} sur $Y$: $\eR i^! \sK\lotimes\sL \to \eR i^! \sM$. 





\subsubsection{}\label{IV:1-2-6}

Ci-dessus, on a déroulé le sorite absolu. On a un sorite relatif 
parallèle, avec $\Gamma$ remplacé par $f_*$ par $f$ un morphisme $X\to S$. 










\subsection{La règle de Koszul}\label{IV:1-3}

Soient $A$ un anneau commutatif, et $(V_i)_{i\in I}$ une famille finie de 
$A$-modules gradués (ou $\dZ/2$-gradués). Rappelons la définition du 
produit tensoriel gradué $\bigotimes_{i\in I} V_i$, au sens de la règle de 
Koszul (cf. \cite[XVII 1.1]{sga4}). Pour chaque ordre total $a$ sur $I$, on va 
définir un module $V(a)$. On va aussi définir un système transitif 
d'isomorphismes $\varphi_{a b}:V(b)\iso V(a)$ et le produit tensoriel 
gradué des $V_i$ sera la ``valeur commune'' $\varprojlim V(a)$ de ce 
système de modules. On prend 
\begin{enumerate}[\indent a)]
  \item $V(a) = \bigotimes_{i\in I} |V_i|$ (produit-tensoriel ordinaire des 
    modules non gradués sous-jacents aux $V_i$).
  \item si les $x_i\in V_i$ sont homogènes, on prend 
    $\varphi_{a b}(\otimes x_i) = (-1)^N \otimes x_i$, où $N$ est la somme 
    des $\deg(x_i)\deg(x_j)$ étendue aux couples $(i,j)$ tels que 
    $i<_a j$ et $i>_b j$.
\end{enumerate}





\subsubsection*{Exemple 1}

Prenons $I=\{1,2\}$ et soient $a$ l'ordre où $1<2$, $b$ l'ordre où $1>2$. 
Soient $v_i\in V_i$, homogènes, et notons $v_1\otimes v_2$ (resp. 
$v_2\otimes v_1$) l'image dans le produit tensoriel gradué du produit des 
$v_i$ dans $V(a)$ (resp. $V(b)$). On a 
\[
  v_1\otimes v_2 = (-1)^{\deg(v_1)\deg(v_2)} v_2\otimes v_1
\]
(règle de Koszul). 





\subsubsection*{Exemple 2}

Si les $V_i$ sont tous de degré $1$, on a , reliant le module sous-jacent au 
produit tensoriel gradué, et le produit tensoriel ordinaire des modules 
$|V_i|$ sous-jacents aux $V_i$, un isomorphisme canonique 
\[
  \left|\bigotimes V_i\right| \simeq \bigotimes |V_i|\otimes_{\dZ}\bigwedge^{|I|}\dZ^I \text{.}
\]





\subsubsection*{Exemple 3}

Pour $A$ un corps, et $X_i$ une famille finie d'espaces, la formule de 
K\"unneth s'écrit 
$\h^\bullet\left(\prod X_i,A\right) = \bigotimes \h^\bullet(X_i,A)$. 

Soit $V^\vee$ le dual gradué de $V$. L'application canonique 
\begin{equation*}\tag{1.3.1}\label{IV:eq:1-3-1}
  V^\vee\otimes V = V\otimes V^\vee \to A
\end{equation*}
est $v'\otimes v\to v'(v)$. 

On suppose maintenant que $A$ est un corps, et on ne considère que des 
espaces vectoriels de dimension finie. L'isomorphisme canonique 
\begin{equation*}\tag{1.3.2}\label{IV:eq:1-3-2}
  W\otimes V^\vee \to \hom(V,W) 
\end{equation*}
est $w\otimes v'\mapsto \left(v\mapsto w\cdot v'(v)\right)$. Via cet 
isomorphisme, la composition $\hom(Y,Z)\otimes \hom(X,Y)\to \hom(X,Z)$ 
s'identifie au morphisme induit par \eqref{IV:eq:1-3-1}: 
$Z\otimes Y^\vee\otimes Y\otimes X^\vee \to Z\otimes X^\vee$. 

La trace de $f:V\to V^\vee$ (nulle pour $f$ homogène de degré $\ne 0$) est 
l'image de $f$ par 
\begin{equation*}\tag{1.3.4}\label{IV:eq:1-3-4}
  \tr:\hom(V,V) {\xleftarrow\sim}_\text{\eqref{IV:eq:1-3-2}} V\otimes V^\vee = V^\vee \otimes V \to_\text{\eqref{IV:eq:1-3-1}} A .
\end{equation*}

On vérifie aisément que, pour $f$ de degré $0$, 
\begin{equation*}\tag{1.3.5}\label{IV:eq:1-3-5}
  \tr(f,V) = \sum (-1)^i \tr(f,V^i) \text{.}
\end{equation*}

Si on exprime que les deux morphismes composés de morphismes 
\eqref{IV:eq:1-3-1} $V^\vee\otimes V\otimes W^\vee\otimes W\to k$ commutent, on 
trouve que, pour $f:V\to W$ et $g:W\to V$ homogènes, on a 
\begin{equation*}\tag{1.3.6}\label{IV:eq:1-3-6}
  \tr(f g) = (-1)^{\deg(f)\deg(g)} \tr(g f) \text{.}
\end{equation*}




















\section{La classe de cohomologie associée à un cycle}\label{IV:2}










\subsection{La classe d'un diviseur}\label{IV:2-1}





\subsubsection{}\label{IV:2-1-1}

Soit $D$ un diviseur de Cartier dans un schéma $X$. Hors de $D$, le faisceau 
inversible $\sO(D)$ est trivialisé par la section $1$. La classe 
$\cl(D)$ de $D$, dans $\h_D^1(X,\dG_m)$, est la classe du 
$\dG_m$-torseur trivialisé sur $X\setminus D$ correspondant 
(\ref{IV:1-1-6} et \ref{IV:1-1-4}). 

Soit $\partial:\h^i(X\setminus D,\dG_m)\to \h_D^i(X,\dG_m)$ le morphisme 
\ref{IV:1-1-4}. Si $D$ admet une équation globale $f$, la multiplication par 
$f$ est un isomorphisme de $\sO(D)$, trivialisé par $1$ sur $X\setminus D$, 
avec $\sO$, trivialisé par $f$ sur $X\setminus D$. D'après \ref{IV:1-1-4}, 
on a donc 
\begin{equation*}\tag{2.1.1}\label{IV:eq:2-1-1}
  \cl(D) = \partial f \text{.}
\end{equation*}

Pour tout morphisme $u:X'\to X$ tel que $u^* D$ soit encore un diviseur de 
Cartier (i.e., $u^{-1} D$ disjoint de $\operatorname{Ass}(X')$), on a 
$\cl(u^* D) = u^*\cl(D)$. Si on voulait une telle 
fonctorialité pour tout morphisme $u$, il faudrait considérer non pas des 
diviseurs de Cartier, mais plus généralement des faisceaux inversibles 
munis d'une section.

Rappelons que l'entier $n$ est dorénavant supposé inversible sur les 
schémas considérés. Soit 
$\partial:\h_D^i(X,\dG_m) \to \h_D^{i+1}(X,\dmu_n)$ le cobord pour la suite 
exacte de Kummer $0 \to \dmu_n \to \dG_m \to \dG_m \to 0$.





\begin{definition}\label{IV:2-1-2}
La \emph{classe $\cl_n(D)$} de $D$ dans $\h_D^2(X,\dmu_n)$ est 
$\partial \cl(D)$. 
\end{definition}

Quand il n'y aura pas de risque de confusion, on omettra la mention de $n$. 





\subsubsection{}\label{IV:2-1-3}

Le diagramme 
\[\xymatrix{
  \h^0(X\setminus D,\dG_m) \ar[r]^-\partial \ar[d]^-\partial 
    & \h_D^1(X,\dG_m) \ar[d]^-\partial \\
  \h^1(X\setminus D,\dmu_n) \ar[r]^-\partial 
    & \h_D^2(X,\dmu_n)
}\]
est anticommutatif. Si $D$ admet une équation globale $f$, 
$\cl_n(D)$ est donc l'opposé de l'image par $\partial$ de la 
classe dans $\h^1(X\setminus D,\dmu_n)$ de $\dmu_n$-torseur des racines 
$n$-ièmes de $f$. 





\begin{proposition}\label{IV:2-1-4}
Soit $i$ l'inclusion de $D$ dans $X$. Si $D$ et $X$ sont réguliers, les 
faisceaux de cohomologie à support $\eR^p i^! \dmu_n$ sont nuls pour $p=0,2$, 
et $\eR^2 i^!\dmu_n = \underline{\dZ/n}$, engendré par 
$\cl_n(D)$. 
\end{proposition}

Il suffit de prouver que, pour $X$ strictement locale et $D$ défini par un 
paramètre régulier, on a $\h_D^p(X,d\mu_n) = 0$ pour $p=0,1$ et 
$\h^2(X,\dmu_n) = \dZ/n$ engendré par $\cl(D)$. Notant par 
$\sim$ la cohomologie réduite, on a 
$\widetilde\h^{p-1}(X\setminus D,\dmu_n)\iso \h_D^p(X,\dmu_n)$. L'assertion 
pour $p=0,1$ exprime que $D$ ne disconnecte par $X$, et pour $p=2$ résulte, 
via \ref{IV:2-1-3}, du lemme d'Abhyankar. 

Ceci est un analogue partiel du théorème relatif 
(\hyperref[I]{Cohomologie étale}, \ref{I:5-3-4}). Grothendieck conjecture 
que les $\eR^p i^! \dmu_n$ sont nuls pour $p\ne 2$, du moins pour $X$ excellent 
(conjecture de pureté), mais ceci n'est connu qu'en caractéristique $0$ 
(\cite[XIX]{sga4}). 




\begin{theorem}[Compatibilité fondamentale]\label{IV:2-1-5}
Soient $X$ uen courbe lisse sur un corps algébriquement clos $k$, $P$ un 
point fermé de $X$ et $\tr$ le composé $\h_P^2(X,\dmu_n) \to \h_c^2(X,\dmu_n) \xrightarrow{\tr} \dZ/n$. On a 
\[
  \tr \cl(P) = 1 \text{.}
\]
\end{theorem}

Soit $\bar X$ la courbe projective et lisse complétant $X$. La formule 
exprime que le faisceau inversible $\sO(P)$ sur $\bar X$ est de degré $1$. 










\subsection{Méthode cohomologique}\label{IV:2-2}





\subsubsection{}\label{IV:2-2-1}

Soit $X$ un schéma (noethérien). Rappelons qu'un sous-schéma $Y$ de $X$ 
est dit d'intersection complète locale, de codimension $c$, si, localement 
(sur $Y$), il est défini par une suite régulière de $c$ équations dans 
$X$. Pour $X$ le spectre d'un anneau locale $A$ d'idéal maximal $\fm$ et 
$Y$ d'idéal $\fa$, cela signifie que $\exp^i(A/\fa,A) = 0$ pour 
$i<\dim(\fa/\fm\fa)=c$, et toute suite d'éléments de $\fa$, d'image dans 
$\fa/\fm\fa$ une base de $\fa/\fm\fa$, est une suite régulière 
d'équations pour $Y$. 





\subsubsection{}\label{IV:2-2-2}

Soit $i:Y\hookrightarrow X$ d'intersection complète locale, de codimension 
$c$. On se propose de définir une classe fondamentale locale 
$\cl(Y)$ qui soit une section globale du faisceau de cohomologie à support 
$\eR^{2c}i^! \dZ/n(c)$ (rappelons que $\dZ/n(c)=\dmu_n^{\otimes c}$). 

Localement, $Y$ est l'intersection d'une suite de $c$ diviseurs $D_i$, et on 
définit $\cl(Y)$ comme le cup-produit des $\cl(D_i)$. Chaque $\cl(D_i)$ est 
à support dans $D_i$, leur produit est à support dans $Y$. Que, localement, 
ce produit ne dépende pas du choix des $D_i$, résulte de \ref{IV:2-2-3} 
ci-dessous et des propriétés d'invariance suivantes:
\begin{enumerate}[\indent a)]
  \item compatibilité à la localisation;
  \item indépendance de l'ordre des $D_i$ (les $\cl(D_i)$ sont de degré 
    $2$, pair, donc le cup-produit est commutatif);
  \item le produit ne dépend que du ``drapeau'' 
    $D_1\supset D_1\cap D_2\supset \cdots\supset Y$. 
\end{enumerate}
Pour prouver c), on note l'existence d'un produit (variante de \ref{IV:1-2-1}) 
\[
  \h_{D_1}^\bullet(X)\otimes \h_{D_1\cap D_2}^\bullet(D_1)\otimes \cdots \otimes \h_Y^\bullet(D_1\cap \cdots \cap D_{c-1}) \to \h_Y^\bullet(X) \text{;}
\]
le produit des $\cl(D_i)$ est encore le produit des ($\cl(D_i)$ restreint à 
$D_1\cap \cdots \cap D_{i-1})=(\cl(D_1\cap \cdots \cap D_i)$ dans 
$D_1\cap \cdots \cap D_{i-1})$. 





\begin{lemma}\label{IV:2-2-3}
Soient $A$ un anneau locale d'idéal maximal $\fm$ et $u=(u_1,\dots,u_c)$, 
$v=(v_1,\dots,v_c)$ deux suites régulières engendrant le même idéal 
$\fa$. Il existe alors une suite $w_i$ ($1\leqslant i\leqslant N$) de telles 
suites, les reliant, telle que $w_{i+1}$ se déduise de $w_i$ par une 
permutation, ou en ne changeant que le dernier élément.
\end{lemma}

Puisqu'on dispose des permutations, il reviendrait au même de se permettre de 
changer un seul élément, plutôt que le dernier. Par \ref{IV:2-2-1}, on se 
ramène alors à vérifier que, dans l'espace vectoriel $\fa/\fm\fa$, on 
peut passer d'une base à une autre par une suite de permutations et de 
changements de base ne modifiant qu'un seul vecteur. Le groupe linéaire est 
en effet engendré par les matrices diagonales, élémentaires et de 
permutation. 





\subsubsection{}\label{IV:2-2-4}

Les mêmes méthodes permettent de définir une classe fondamentale locale 
pour tout $Y\subset X$ localement définissable par $c$ équations (la où 
moins d'équations suffisent, la classe est nulle). 





\subsubsection{}\label{IV:2-2-5}

On peut passer d'une telle classe fondamentale locale, dans 
$\h^0\left(Y,\eR^{2 c} i^!\dZ/n(c)\right)$, à une classe fondamentale globale, 
dans $\h_Y^{2 c}\left(X,\dZ/n(c)\right)$, lorsqu'on dispose de résultats de 
semi-pureté:





\begin{proposition}\label{IV:2-2-6}
\leavevmode
\begin{enumerate}[(i)]
  \item Si $\eR^p i^!\dZ/n=0$ pour $p<2 c$, alors 
    \[
      \h_Y^{2 c}\left(X,\dZ/n(c)\right) \iso \h^0\left(Y,\eR^{2 c} i^! \dZ/n(c)\right) \text{.}
    \]
  \item Soit $Z$ une partie fermée de $Y$, de complément $V$ dans $Y$, et 
    $k$ l'inclusion de $Z$ dans $X$. Si $\eR^p i^! \dZ/n = 0$ pour 
    $p\leqslant 2 c$, alors 
    \[
      \h_Y^{2 c}\left(X,\dZ/n(c)\right) \hookrightarrow \h_V^{2 c}\left(X,\dZ/n(c)\right) \text{.}
    \]
    Si $\eR^p i^!\dZ/n=0$ pour $p\leqslant 2 c+1$, cette flèche est une 
    isomorphisme.
\end{enumerate}
\end{proposition}

L'hypothèse $\eR^p i^!\dZ/n=0$ équivaut à $\eR^p i^!\dZ/n(c)=0$. Ceci dit, 
(i) se lit sur la suite spectrale 
$\h^p(Y,\eR^q i^!)\Rightarrow \h_Y^{p+q}(X,-)$. Si $k_1$ l'inclusion de $Z$ dans 
$Y$, on a $k=i k_1$, d'où $\eR k^! \dZ/n=\eR k_1^! \eR i^!$, et la suite exacte 
longue de cohomologie pour $Z\subset Y$ fournit une suite exacte longue 
\[\xymatrix{
  \cdots \ar[r] 
    & \h^i\left(Z,\eR k^! \dZ/n(c)\right) \ar[r] 
    & \h_Y^i\left(X,\dZ/n(c)\right) \ar[r] 
    & \h_V^i\left(X,\dZ/n(c)\right) \ar[r] 
    & \cdots
}\]
dont (ii) résulte. 





\subsubsection{Amplification}\label{IV:2-2-7}

La proposition \ref{IV:2-2-6} reste valable pour $i$ un quelconque morphisme 
séparé de type fini, et $2 c$ un entier (positif ou négatif) quelconque, 
pour autant qu'on y remplace $\h_Y^{2 c}\left(X,\dZ/n(c)\right)$ par 
$\h^{2 c}\left(Y,\eR i^!\dZ/n(c)\right)$, et de même pour $\h_V$. 

Les résultats de semi-pureté suivants résultent de 
\cite[1.8, 1.10, 1.15]{sga2}. Nous rappellerons leur preuve. 





\begin{theorem}\label{IV:2-2-8}
Soit un morphisme de $S$-schémas de type fini 
\[\xymatrix{
  Y \ar[r]^-i \ar[dr]_-f 
    & X \ar[d]^-g \\
  & S
}\]
On suppose $X$ lisse, purement de dimension relative $N$ et que $Y$ est fibre 
par fibre de dimension $\leqslant d$. Soit $c=N-d$. 
\begin{enumerate}[\indent (i)]
  \item On a $\eR^p i^!\dZ/n=0$ pour $p<2 c$; de même pour $\dZ/n$ remplacé 
    par un faisceau $g^* \sF$.
  \item Si, au dessus d'un ouvert dense $U$ de $S$, $Y$ est fibre par fibre de 
    dimension $<d$, on a $\eR^p i^!\dZ/n=0$ pour $p\leqslant 2 c$. Si de plus le 
    complément $Z$ de $U$ ne disconnecte pas localement $S$, on a 
    $\eR^pi^!\dZ/n=0$ pour $p\leqslant 2 c+1$.
\end{enumerate}
\end{theorem}

Puisque $\eR g^! \sF=g^*\sF(N)[2N]$, la formule de transitivité 
$\eR i^!\eR g^!=\eR f^!$ montre que 
$\eR^{2 c+q} i^!(g^*\sF)=0\Leftrightarrow \eR^{-2 d+q}f^!(\sF)$. Ceci nous 
ramène à étudier $f$, i.e. à supposer que $X=S$. L'assertion (i) est 
alors \cite[XVIII 3.17]{sga4}. 

Soit $Y'$ l'image inverse de $Z$ dans $Y$:
\[\xymatrix{
  Y' \ar[r]^-v \ar[d]^-f 
    & Y \ar[d]^-f \\
  Z \ar[r]^-u 
    & S
}\]
D'après (i), les $\eR^p f^! \dZ/n$ sont à support dans $Y'$ pour 
$p\leqslant -2 d+1$; et la suite spectrale 
$\eR^a v^!\eR^b f^! \Rightarrow \eR^{a+b} (f v)^!$ montre qu'ils coïncident avec 
les $\eR^p(f v)^!\dZ/n=\eR^p(u f)^!\dZ/n$. Appliquant (i) à $Y'/Z$ et à la 
suite spectrale $\eR^a f^!\eR^b u^! \Rightarrow \eR^{a+b} (u f)^!$, on trouve que 
(ii) résulte de la nullité de $\eR^b u^! \dZ/n$ pour $b=0$, ou $b=0$ et $1$ 
selon le cas. 





\subsubsection{}\label{IV:2-2-9}

Grothendieck conjecture l'analogue absolu suivant de \ref{IV:2-2-8} (conjecture 
de semi-pureté -- une conséquence de la conjecture de pureté): pour $Y$ 
de codimension $\geqslant c$ dans $X$ régulier, on a $\h_Y^i(X,\dZ/n)=0$ pour 
$i<2 c$, tout au moins si $X$ est excellent. 





\subsubsection{}\label{IV:2-2-10}

On est maintenant à pied d'œuvre pour définir la classe d'un cycle $Y$ de 
codimension $c$ dans $X$ lisse sur un corps $k$. On écrit $Y=\sum d_i Y_i$, 
où les $Y_i$ sont réduits irréductibles. Un ouvert $U_i$ de $Y_i$, de 
complémentaire de codimension $>c$, est alors d'intersection complète 
locale dans $X$. Ceci permet de définir la classe fondamentale locale de 
$U_i$. D'après \ref{IV:2-2-6} et \ref{IV:2-2-8}, celle-ci provient d'une 
unique classe fondamentale $\cl(Y_i)\in \h_{Y_i}^{2 c}\left(X,\dZ/n(c)\right)$, 
et on pose 
\[
  \cl(Y) = \sum d_i \cl(Y_i) \in \h_{|Y|}^{2 c}\left(X,\dZ/n(c)\right) \text{.}
\]





\subsubsection{}\label{IV:2-2-11}

En géométrie analytique complexe, une construction de Baum, Fulton et 
Mac Pherson \cite{bfm75} permet de définir sans restriction la classe de 
cohomologie d'une intersection complète locale $Y\subset X$. Supposons $Y$ 
purement de codimension $c$, et soit $\sN$ le fibré vectoriel sur $Y$ fibré 
normal de $Y$ dans $X$. Ses sections locales sont celles de $(\sI/\sI^2)^\vee$, 
d'où $\sI$ est le faisceau d'idéaux de $Y$. Rappelons que le faisceau des 
fonctions $C^\infty$ sur $X$ est défini localement, en terme de plongements 
locaux de $X$ dans $\dC^n$, comme la restriction à $X$ du quotient du 
faisceaux des fonctions $C^\infty$ sur $\dC^n$ par l'idéal engendré par les 
parties réelles et imaginaires des équations qui définissent $X$. Le 
fibré $\sN$ se prolonge en un fibré vectoriel complexe $C^\infty N$ sur un 
voisinage $U$ de $Y$ dans $X$, et pour $U$ assez petit, il existe des sections 
$f$ de $N$, de lieu des zéros $Y$, et telles que, sur $Y$, $d f:\sN\to N$ 
soit l'identité. Deux choix de $N$ et $f$ sont homotopes sur $Y$ assez petit. 

La classe de cohomologie $\cl(U)$ de la section $0$ de $N$ (notée $z:U\to N$) 
est définie: $U$ est d'intersection complète locale dans $N$, et 
$\eR^i z^!\dZ = 0$ pour $i\ne 2 c$, $\eR^{2 c} z^!\dZ = \dZ$. On dispose de 
$f^*:\h_U^\bullet(N,\dZ) \to \h_Y^\bullet(X,\dZ)$ et on pose 
\[
  \cl(Y) = f^*\cl(U) \text{.}
\]

La même construction marche dès qu'on a sur un sous-espace analytique $Y$ 
de $X$ la structure normale suivante: un faisceau localement libre $\sC$ de 
rang $c$ sur $Y_\text{red}$, et un épimorphisme 
$\sC\to \sI/\sI\cdot \sI_\text{red}$. 










\subsection{Méthode homologique}\label{IV:2-3}





\subsubsection{}\label{IV:2-3-1}

Soit $f:Y\to X$ un morphisme plat de type fini, à fibres de dimension 
$\leqslant d$. Dans \cite[XVIII 2.9]{sga4}, nous avons défini un morphisme 
trace 
\begin{equation*}\tag{2.3.1.1}\label{IV:eq:2-3-1-1}
  \tr_f:\eR^{2 d} f_! \dZ/n(d) \to \dZ/n \text{.}
\end{equation*}
On a $\eR^i f_!\dZ/n(d)=0$ pour $i>2 d$, et $\eR^i f^!\dZ/n=0$ pour $i<-2 d$. 
Ceci, et l'adjonction entre $\eR f_!$ et $\eR f^!$, fournissent des isomorphismes 
\begin{equation*}\tag{2.1.3.2}\label{IV:eq:2-1-3-2}
\begin{aligned}
  \hom\left(\eR^{2 d}f_!\dZ/n(d),\dZ/n\right) 
    &= \hom\left(\eR f_!\dZ/n(d),\dZ/n[-2 d]\right) \\
    &= \hom\left(\dZ/n,\eR f^!\dZ/n(-d)[-2 d]\right) \\
    &= \h^0\left(Y,\eR^{-2 d}f^!\dZ/n\right) \text{.}
\end{aligned}
\end{equation*}
Au moins sur $S$ un point, l'image de $\tr_f$ dans les deux derniers groupes 
mérite le nom de classe fondamentale de $Y$ (en homologie). 

On note encore $\tr_f$ l'image de $\tr_f$ dans le second groupe, et les 
morphismes qui s'en déduisent par fonctorialité. Par exemple, pour $Y/S$ 
propre, le morphisme 
\begin{equation*}\tag{2.3.1.3}\label{IV:eq:2-3-1-3}
\xymatrix{
  \h^i\left(Y,\dZ/n(d)\right) = \h^i\left(S,\eR f_!\dZ/n(d)\right) \ar[r] 
    & \h^{i-2 d}(S,\dZ/n) \text{.}
}
\end{equation*}

Supposons $Y$ contenu dans $X$ lisse sur $S$, purement de dimension relative 
$N$, et posons $c=N-d$. 
\[\xymatrix{
  Y \ar@{^{(}->}[r]^-i \ar[dr]_-f 
    & X \ar[d] \\
  & S
}\]
On a $\eR f^! = \eR i^!\eR g^!$, et $\eR g^! \dZ/n=\dZ/n(-N)[-2 N]$. La classe 
fondamentale de $Y$ s'identifie à un élément de 
$\h^0\left(Y,\eR i^!\dZ/n(c)[2 c]\right)=\h_Y^{2 c}\left(X,\dZ/n(c)\right)$, la 
classe $\cl(Y)$ de $Y$ dans $X$. Nous verrons plus loin qu'elle ne dépend que 
de $Y\subset X$, non de la projection de $X$ sur $S$, et que pour $Y$ 
d'intersection complète locale, elle induit la classe locale du numéro 
précédent. 

Explicitons l'isomorphisme 
\[
  \h_Y^{2 c}\left(X,\dZ/n(c)\right) \iso \hom\left(\eR^{2 d}f_!\dZ/n(d),\dZ/n\right)
\]
qui transforme $\cl(Y)$ en $\tr_f$: via les isomorphismes 
\begin{align*}
  \h_Y^{2 c}\left(X,\dZ/n(c)\right) 
    &= \h_Y^{2 c}\left(\eR i^!\dZ/n(c)\right) \\
    &= \hom_Y\left(\dZ/n(d)[2 d],\eR i^!\dZ/n(N)[2 N]\right) \\
    &= \hom_X\left(\dZ/n(d)[2 d]_Y,\dZ/n(N)[2 n]\right) \text{,}
\end{align*}
c'est la ligne supérieure de 
\[\xymatrix{
  \h_Y^{2 c} \ar[r] \ar[dr]_-{(3)} 
    & \hom\left(\eR f_!\dZ/n(d)[2 d],\eR f_! \eR i^!\dZ/n(N)[2 n]\right) \ar[r]^-{(1)} \ar[d]^-{\tr_i} \ar@{}[dr]|-{(2)}
    & \hom\left(\eR f_!\dZ/n(d)[2d],\dZ/n\right) \ar@{=}[d] \\
  & \hom\left(\eR g_!\dZ/n(d)[2 d]_Y,\eR g_!\dZ/n(N)[2 N]\right) \ar[r]^-{\tr_g} 
    & \hom\left(\eR g_!\dZ/n(d)[2 d]_Y,\dZ/n\right) \text{,}
}\]
où (1) est la flèche d'adjonction $\eR f_!\eR f^!\to \text{id}$. La 
commutativité (2) exprime que l'isomorphisme $\eR f^!=\eR i^!\eR g^!$ est 
défini par adjonction. Enfin, la flèche déduite de (3)
\[
  \h_Y^{2c}\left(X,\dZ/n(c)\right) \otimes \eR^{2 d} g_!\dZ/n(d) \to \eR^{2 N} g_!\dZ/n(N)
\]
s'interprète comme un cup-produit (cf. \ref{IV:1-2}). 





\begin{definition}\label{IV:2-3-2}
La classe $\cl(Y)\in\h_Y^{2 c}\left(X,\dZ/n(c)\right)$ a pour propriété 
caractéristique que, pour tout section locale $u$ de $\eR^{2 d}f_! \dZ/n(d)$, 
on a 
\[
  \tr_f(u) = \tr_g\left(\cl(Y)\smallsmile u\right) \text{.}
\]
\end{definition}

Du fait que $\cl(Y)$ ait déjà une image $\tr_f$ dans 
$\hom\left(\eR f_!\dZ/n(d)[2 d],\dZ/n\right)$, le formule \ref{IV:2-3-2} vaut 
pour les flèches déduites de $\tr_f$ par fonctorialité. Par exemple, 
pour $X$ sur $S$ propre et $u\in \h^\bullet(Y,\dZ/n)$, la formule vaut dans 
$\h^\bullet\left(S,\dZ/n(-d)\right)$. Pour $v\in \h^\bullet(X,\dZ/n)$, elle 
donne 
\[
  \tr_f(i^* v) = \tr_g\left(\cl(Y)\smallsmile v\right) \text{,}
\]
où le cup-produit peut se calculer dans $\h^\bullet(X,\dZ/n)$. 





\subsubsection{}\label{IV:2-3-3}

Nous allons définir les morphisme trace, et donc la classe $\cl(Y)$, sous des 
hypothèses plus générales. Soit donc un diagramme 
\[\xymatrix{
  Y \ar@{^{(}->}[r] \ar[dr]_-f 
    & X \ar[d]^-g \\
  & S
}\]
avec $g$ lisse, purement de dimension relative $N$ et $Y$ fermé dans $X$, 
fibre par fibre de dimension $\leqslant d$. On munit de plus $Y$ d'un 
``poids'' $\sK$ du type suivant: $\sK\in\eD_{\text{parf},X}^b$ est un 
complexe borné de faisceaux de $\sO$-modules sur $X$, de $\tor$-dimension 
finie sur $S$ (ou sur $X$, cela revient au même) et dont les faisceaux de 
cohomologie soient cohérents et à support dans $Y$. On se propose de 
définir un morphisme $\tr_{f,\sK}:\eR^{2 d}f_! \dZ/n(d) \to \dZ/n$. Pour $Y$ 
plat sur $S$ et $\sK=\sO_Y$, ce sera le morphisme trace précédent. En 
général, il ne dépend que des longueurs de $\sK$ aux points 
génériques $y$ des composantes irréductibles de 
$Y(\operatorname{lg}_y(\sK) = \sum (-1)^i \operatorname{lg} \h^i(\sK)_y)$. 
% NOTE: notation is not clear, is it a calligraphic H?
Enfin, on notera $\cl(Y,\sK)$ la classe à support dans $Y$ telle que 
\begin{equation*}\tag{2.3.3.1}\label{IV:eq:2-3-3-1}
  \tr_g\left(\cl(Y,\sK)\smallsmile u\right) = \tr_{f,\sK}(u) \text{.}
\end{equation*}





\subsubsection{}\label{IV:2-3-4}

La construction de $\tr_{f,\sK}$ est parallèle à celle de 
\cite[XVIII.2]{sga4}; nour n'en indiquerons que les grandes lignes.

\paragraph{A}
$d=0$ ($f$ quasi-fini). Soit $x$ un point géométrique de $Y$. $s$ son image 
dans $S$, $\sK_{(x)}$ l'image réciproque de $\sK$ sur le localisé stricte 
de $X$ en $x$, $\sK_{(x)s} = \sK_{(x)}\lotimes_{\sO_{S,s}^h} k(s)$ son image 
réciproque sur la fibre géométrique en $s$, et 
$n(x)=\sum (-1)^i \dim_{k(s)} \h^i(\sK_{(x)s})$. La fonction $x\mapsto n(x)$ 
est une pondération de $f$, et on prend le morphisme trace correspondant 
\cite[XVII.6.2.5]{sga4}. La pondération $n(x)$, et $\tr_{f,\sK}$ sont de 
formation compatible à tout changement de base. 

\paragraph{B}
Cas général. Si $u:Z\to \dA_S^d$ est tel que $u i$ soit quasi-fini, on pose 
$\tr_{f,\sK} = \tr_{\dA_S^d}\circ \tr_{u i,\sK}$. Ce morphisme trace est 
clairement compatible à tout changement de base $S'/S$. Pour prouver qu'il ne 
dépend pas de $u$, on peut donc supposer que $S$ est le spectre d'un corps 
algébriquement clos $k$. Soient dans ce cas $Y_i$ les composantes 
irréductibles de $Y$, et $\operatorname{lg}_i$ la longueur de $\sK$ au 
point générique de $Y_i$. Si $\alpha_i$ est l'inclusion de 
$(Y_i)_\text{red}$ dans $Y$, on a 
\begin{equation*}\tag{2.3.4}\label{IV:eq:2-3-4}
  \tr_{f,\sK} = \sum \operatorname{lg}_i \tr_{Y_{i,\text{red}/S}} \alpha_i^\ast
\end{equation*}
(cette formule résulte de la formule analogue et facile pour $\tr_{u i}$). 

Ceci définit $\tr_{f,\sK}$ localement sur $Y$; on procède ensuite comme en 
\cite[XVIII.2.9]{sga4}. 





\begin{lemma}\label{IV:2-3-5}
Si $g=g'g'':X\xrightarrow{g''} S'\xrightarrow{g'} S$, avec $g'$ et $g''$ lisse 
et purement de dimension relative $N'$ et $N''$, et que $Y$ est encore, sur 
$S'$, fibre par fibre de codimension $\geqslant c$, alors la classe 
$\operatorname{cl}(Y,\sK)$ est la même, calculée en terme de $g$ ou de 
$g''$.
\end{lemma}

Soient $f'=g'' i$ et $d'=d-N'$. La formule $\tr_g=\tr_{g'}\tr_{g''}$ 
(\cite[XVIII.2.9]{sga4} Var 3) assure la commutativité de 
\[\xymatrix{
  \h_X^{2 c}\left(Z,\dZ/n(c)\right) \ar[d] \\
  \hom_{S'}\left(\eR f_!\dZ/n(d')[2 d'],\eR g_!''\dZ/n(N'')[2 N'']\right) \ar[r] \ar[d]^-{\eR g_!'(N')} 
    & \hom_{S'}\left(\eR^{2 d'} f_!' \dZ/n,\dZ/n\right) \ar[d]^-{\tr_{g'}\circ \eR^{2 N'} g_!'} \\
  \hom_S\left(\eR f_!\dZ/n(d)[2d],\eR g_!'\dZ/n(N)[2 N]\right) \ar[r] 
    & \hom_S\left(\eR^{2 d} f_!\dZ/n,\dZ/n\right)
}\]
tandis que la formule analogue et facile à vérifier 
$\tr_{f,\sK}=\tr_{g'}\tr_{f',\sK}$ assure que l'image de $\tr_{f',\sK}$ est 
$\tr_{f,\sK}$. L'assertion en résulte. 





\begin{lemma}\label{IV:2-3-6}
Si $Y$ est un diviseur dans $X$, la classe \ref{IV:2-3-2} coïncide avec la 
classe \ref{IV:2-1-2}. 
\end{lemma}

Par semi-pureté (\ref{IV:2-2-6}(i) et \ref{IV:2-2-8}(i)), le problème est 
locale sur $Y$; utilisant \ref{IV:2-3-5} pour remplacer $S$ par $S'$ 
convenable, ceci nous permet de supposer que $N=1$. Utilisant à nouveau la 
semi-pureté (\ref{IV:2-2-6}(ii) et \ref{IV:2-2-8}(ii)) il suffit de prouver 
\ref{IV:2-3-6} au-dessus des points génériques de $S$. Une localisation 
étale nous ramène alors à supposer $S$ spectre d'un corps 
algébriquement clos $k$. Les classes \ref{IV:2-3-2} et \ref{IV:2-1-2} étant 
chacune additive en $Y$, ceci nous ramène au cas où $Y$ est un point 
fermé sur un courbe lisse sur $k$. Que la propriété caractéristique 
\ref{IV:2-3-2} soit vérifiée est alors la compatibilité fondamentale 
\ref{IV:2-1-5}. 





\begin{lemma}\label{IV:2-3-7}
La formation de $\cl(Y,\sK)$ est compatible à tout changement de base 
$S'/S$.
\end{lemma}

Résulte de la même assertion pour les morphismes trace.





\begin{theorem}\label{IV:2-3-8}
(i) $\cl(Y,\sK)$ ne dépend que de $Y\subset X$ et des longueurs de $\sK$ aux 
points génériques de $Y$. Cette classe est additive en $\sK$. Pour 
$\sK=\sO_Y$ et $Y$ d'intersection complète locale, elle induit la classe 
locale \ref{IV:2-2-2}.

(ii) Soit un diagramme commutatif 
\[\xymatrix{
  X' \ar[r]^-u \ar[d] 
    & X \ar[d] \\
  S' \ar[r] 
    & S
}\]
avec $X'$ lisse sur $S'$. Si $u^{-1}(Y)$ est encore fibre par fibre de 
codimension $\geqslant c$, alors 
\[
  \cl\left(u^{-1}(Y),\eL u^\ast\sK\right) = u^\ast \cl(Y,\sK) \text{.}
\]

(iii) Soit $Y'\subset X$ fibre par fibre de codimension $\geqslant c'$, 
$\sK'$ un poids sur $Y'$ et supposons que $Y\cap Y'$ soit fibre par fibre de 
codimension $\geqslant c+c'$. Alors 
\[
\cl\left(Y\cap Y',\sK\lotimes \sK'\right) = \cl(Y,\sK) \smallsmile \cl(Y',\sK') \text{.}
\]
\end{theorem}


% The following paragraphs were originally numbered

% (2.3.8.1)
\paragraph{(A)}
Etant donné $Y\subset X/S$ comme dit, il résulte de la semi-pureté 
(\ref{IV:2-2-6}, \ref{IV:2-2-8}) que pour vérifier que deux classes dans 
$\h_Y^{2 c}\left(X,\dZ/n(c)\right)$ coïncident, il suffit de le vérifier 
localement aux points génériques de $Y$, voire même après tout 
changement de bas $s\to S$ ($s$ point géométrique générique de $S$), 
aux points génériques de $Y$. 

% (2.3.8.2)
\paragraph{(B) Preuve de (ii) pour $S=S'$ et $u$ un plongement fermé}
Par localisation (A) on peut supposer que $X'$ est l'image 
réciproque de la section $0$ par un morphisme lisse $v:X\to \dA_S^{N'}$ 
\[\xymatrix{
  X' \ar[r]^-u \ar[d] 
    & X \ar[d]^-v \\
  S \ar[r]^-0 
    & \dA_S^{N'} \ar[r]^-\pi 
    & S
}\]
On applique \ref{IV:2-3-5} à $g=\pi v$ et \ref{IV:2-3-7} au changement de 
base $S\to \dA_S^{N'}$. 

% (2.3.8.3)
\paragraph{(C) Dans une situation produit:}
$X=X'\times_S X''$, $Y=Y'\times_S Y''$, $\sK=\sK\lotimes_S \sK''$, on a 
$\cl(Y,\sK) = \cl(Y,\sK')\smallsmile \cl(Y'',\sK'')$ (cup-produit extérieur, 
i.e. $\operatorname{pr}_1^\ast \cl(Y',\sK')\smallsmile \operatorname{pr}_2^\ast \cl(Y'',\sK'')$). 

Des propriétés fonctorielles de $\tr_g$ \cite[XVIII.2.12]{sga4} résulte 
la commutativité de 
\small
\[\xymatrix@=0.5cm{
  \h_{X'}^{2c'}\otimes \h_{X''}^{2 c''}\ar[r] \ar[d] 
    & \hom_S\left(\eR f_!' \dZ/n(d')[2 d'],\eR g_!'\dZ/n(N')[2 N']\right)\otimes \cdots \ar[r] \ar[d] 
    & \hom_S\left(\eR^{2 d'} f_!' \dZ/n,\dZ/n\right) \otimes \cdots \ar[d] \\
  \h_X^{2 c} \ar[r] 
    & \hom_S\left(\eR f_!\dZ/n(d)[2 d],\eR g_!\dZ/n(N)[2 N]\right) \ar[r] 
    & \hom_S\left(\eR^{2 d}f_!\dZ/n,\dZ/n\right)
}\]
\normalsize
et on vérifie que $\tr_f$ est l'image de $\tr_{f'}\otimes \tr_{f''}$ d'où 
l'assertion. 

\paragraph{Preuve de (ii)}
(\ref{IV:2-3-7}) et un changement de base préliminaire nous ramènent à 
supposer que $S=S'$. Factorisons $u$ par son graphe:
\[\xymatrix{
  X' \ar[r]^-{(\text{id},u)} 
    & X'\times X \ar[r]^-{\operatorname{pr}_2} 
    & X \text{.}
}\]
Ceci nous amène à traiter séparément $\operatorname{pr}_2$, justiciable 
de (C) pour $X'=X'$, $\sK'=\sO_{x'}$ (avec $\cl(Y')=1\in \h^0(X,\dZ/n)$), et 
$(\text{id},u)$, justiciable de (B). 

\paragraph{Preuve de (iii)}
Soit $\delta:X\to X\times_S X$ l'inclusion diagonale. On utilise la formule 
$\delta^\ast(Y_1\times_S Y_2) = Y_1\cap Y_2$ pour se ramener à 
(B) et (C). 

\paragraph{Preuve de (i)}
Le changement de base $S_\text{red}\to S$ remplace $X$ par $X_\text{red}$; ceci 
nous ramène à supposer $X$ et $S$ réduits. Une localisation au voisinage 
des points génériques de $Y$ nour ramène ensuite au cas où 
$Y_\text{red}$ est irréductibles et l'intersection complète de $c$ 
diviseurs $D_i$. La même localisation, et la définition de $\tr$ dans le 
cas quasi-fini. montre ensuite que 
$\cl(Y,\sK)=\operatorname{lg}\cl(Y_\text{red},\sO_{Y_\text{red}})$, où 
$\operatorname{lg}$ est la longueur de $\sK$ au point générique de $Y$. 
D'après (iv), on a donc 
\[
  \cl(Y,\sK) = \operatorname{lg} \prod_i \cl(D_i,\sO_{D_i})
\]
et on conclut par \ref{IV:2-3-6}. 





\subsubsection{Remarque}\label{IV:2-3-9}

Pour $S$ le spectre d'un corps, et la classe d'un cycle étant définie comme 
en \ref{IV:2-2-10}, l'assertion (iii) dit que, si deux cycles $Y'$ et $Y''$ se 
coupent avec la bonne dimension, alors 
\[
  \cl(Y'\cap Y'') = \cl(Y')\smallsmile \cl(Y'')\text{,}
\]
pourvu que les multiplicités des composantes de $Y'\cap Y''$ soient 
calculées comme sommes alternées de $\tor$. 





\subsubsection{Remarque}\label{IV:2-3-10}

Soit $X/\spec(k)$, $k$ algébriquement clos. Si deux cycles de codimension $c$ 
dans $X$ sont algébriquement équivalents, il résulte de l'existence de la 
théorie relative ci-dessus que les images dans $\h^{2 c}(X,\dZ/n(c))$ de 
leurs classes sont les fibres, en deux points de $S$ connexe convenable, d'une 
section sur $S$ du faisceau constant $\h^{2 c}(X,\dZ/n(c))$ (le faisceau 
$\eR^{2 c} f_\ast \dZ/n(c)$ pour $f=\operatorname{pr}_2:X\times S\to S$): deux 
cycles algébriquement équivalentes ont même classe dans 
$\h^{2 c}(X,\dZ/n(c))$. 




















\section{Application: la formule des traces de Lefschetz dans le cas propre et lisse}\label{IV:3}





\subsection{}\label{IV:3-1}

Soient $X$ et $Y$ des variétés algébriques propres et lisses sur un corps 
algébriquement clos $k$, purement de dimensions $N$ et $M$. Via 
l'isomorphisme de K\"unneth 
$\h^\bullet(X\times Y)=\h^\bullet(X)\otimes \h^\bullet(Y)$, le morphisme trace 
$\h^\bullet(X\times Y)(M)\to \h^\bullet(X)$ n'est autre que 
$\h^\bullet(X)\otimes ($le morphisme trace $\h^\bullet(Y)(M)\to \dQ_\ell)$. Le 
morphisme étant homogène de degré pair, il n'y a pas de problème de 
signe. On le note $\int_Y$. 





\subsection{}\label{IV:3-2}

Une classe $\eta\in \h^{2 q}(X\times Y)(q)$ définit un morphisme de degré 
$2(q-N)$ $\eta_{XY}:\h^\bullet(Y)(N-q)\to \h^\bullet(X)$, par la formule 
\[
  \beta \mapsto \int_Y \eta\cdot \operatorname{pr}_2^\ast \beta \text{.}
\]





\begin{proposition_}\label{IV:3-3}
Soient $p$ et $q$ deux entiers, avec $p+q=N+M$, et 
$\varepsilon\in \h^{2 p}(X\times Y)(p)$, $\eta\in \h^{2 q}(X\times Y)(p)$. La 
trace de $\eta_{X Y}\varepsilon_{Y X}:\h^\bullet(X)\to \h^\bullet(X)$ étant 
entendue au sens \ref{IV:1-3}, on a alors 
\[
  \tr_{X\times Y}(\eta\cdot \varepsilon) = \tr\left(\eta_{X Y}\varepsilon_{Y X},\h^\bullet(X)\right) \text{.}
\]
\end{proposition_}

Pour exorciser les signes, il y a intérêt à ne retenir de la graduation 
de $\h^\bullet$ que la $\dZ/2$-graduation sous-jacente. Pour éviter de 
trainer avec soi les twist à la Tate, on fixe de plus un isomorphisme 
$\dQ_\ell(1)=\dQ_\ell$. Soit $\alpha_Y$ l'homomorphisme 
$\h^\bullet(Y)\to \h^\bullet(Y)^\vee$ (un isomorphisme) rendant commutatif le 
diagramme 
\[\xymatrix{
  \h^\bullet(Y)\otimes \h^\bullet(Y) \ar[r] \ar[d]^-{\alpha_Y\otimes 1} 
    & \h^\bullet(Y) \ar[r]^-{\tr} 
    & \dQ_\ell \ar@{=}[d] \\
  \h^\bullet(Y)^\vee \otimes \h^\bullet(Y) \ar[rr]^-{\text{(\ref{IV:eq:1-3-1})}} 
    & & \dQ_\ell \text{.}
}\]
L'application $\eta_{X Y}$ est l'image de $\eta$ par 
\[\xymatrix{
  \h^\bullet(X\times Y) = \h^\bullet(X)\otimes \h^\bullet(Y) \ar[r]^-{\alpha_Y} 
    & \h^\bullet(X) \otimes \h^\bullet(Y)^\vee \ar@{=}[r]^-{\text{(\ref{IV:eq:1-3-2})}} 
    & \hom\left(\h^\bullet(X),\h^\bullet(Y)\right) \text{,}
}\]
et de même pour $\varepsilon$. La définition \ref{IV:eq:1-3-4} de la trace 
ramène alors \ref{IV:3-3} à la formule 
$\tr_{X\times Y}=\tr_X\otimes \tr_Y$. 





\subsection{Remarque}\label{IV:3-4}

La preuve de \ref{IV:3-4} n'utilise pas que $\alpha_X$ et $\alpha_Y$ soient des 
isomorphismes. Elle vaut sans supposer $X$ et $Y$ lisses. Toutefois, en dehors 
du cas lisse, il est difficile de trouver des classes de \emph{cohomologie} 
auxquelles on veuille appliquer \ref{IV:3-3}. 





\subsection{Remarque}\label{IV:3-5}

Si $\varepsilon,\eta$ sont les classes de cohomologie de cycles algébriques 
$A$ et $B$ sur $X\times Y$, de codimension $p$ et $q$, et que $A\cap B$ est de 
dimension $0$, alors $\tr_{X\times Y}(\eta\varepsilon)$ est la multiplicité 
d'intersection $A\cdot B$, calculée par une somme alternée de $\tor$ 
(\ref{IV:2-3-9}). 





\begin{proposition_}\label{IV:3-6}
Pour $q=M$ et $\eta=\cl(B)$ la classe d'un sous-schéma $B\subset X\times Y$, 
fini sur $X$, l'application $\eta_{X Y}$ est la composée 
\[\xymatrix{
  \eta_{X Y} : \h^\bullet(Y) \ar[r]^-{\operatorname{pr}_2^\ast} 
    & \h^\bullet(B) \ar[r]^-{\tr_{B/X}} 
    & \h^\bullet(X) \text{.}
}\]
\end{proposition_}

D'après \ref{IV:3-2} et \eqref{IV:eq:2-3-3-1} pour $\sK=\sO_B$, on a en 
effet 
\[
  \eta_{X Y}(\beta) = \tr_{X\times Y/X}\left(\cl(B)\cdot \operatorname{pr}_2^\ast\beta\right) = \tr_{B/X}\left(\operatorname{pr}_2^\ast \beta\right) \text{.}
\]

Le cas le plus important est celui où $B$ est le graphe d'une application 
$f:X\to Y$. Dans ce cas, \ref{IV:3-6} montre que $\eta_{X Y}=f^\ast$. 





\begin{corollary_}\label{IV:3-7}
Soit $f$ un endomorphisme de $X$ propre et lisse sur un corps algébriquement 
clos $k$. On suppose que les points fixes de $f$ sont isolés. Alors, la 
trace $\sum (-1)^i \tr(f^\ast,\h^i(X))$ est le nombre de points fixes de $f$, 
chacun compté avec sa multiplicité.
\end{corollary_}

Dans \ref{IV:3-5}, on prend $A=$graphe de l'identité, $B=$graphe de $f$ et on 
applique \ref{IV:3-3}, \ref{IV:3-6}. 





\begin{corollary_}\label{IV:3-8}
Soient $X$ un courbe sur $k$ algébriquement clos, déduit par extension des 
scalaires de $X_0$ sur $\dF_q$, et $f$ le morphisme de Frobenius. Alors, 
$\sum (-1)^i \tr\left(f^\ast,\h_c^\bullet(X)\right)$ est le nombre de points 
fixes de $f$.
\end{corollary_}

On peut remplacer $X$ par $X_\text{red}$, donc supposer $X$ réduite. Soit $U$ 
l'ouvert de $X$ où $X$ est lisse de dimension $1$, et $\bar U$ le 
complétion projective et lisse de $U$. Les suites exactes longues de 
cohomologie pour $(X,U)$ et $(\bar U,U)$ donnent 
\[
  \tr\left(f^\ast,\h_c^\bullet(X)\right) = \tr\left(f^\ast,\h_c^\bullet(U)\right) + \tr\left(f^\ast,\h_c^\bullet(X\setminus U)\right) 
\]
et
\[
  \tr\left(f^\ast,\h_c^\bullet(\bar U)\right) = \tr\left(f^\ast,\h_c^\bullet(U)\right) + \tr\left(f^\ast,\h_c^\bullet(\bar U\setminus U)\right) \text{.}
\]
Le mêmes formules valent pour les nombres de points fixes. La formule 
\ref{IV:3-8} étant claire pour un schéma de dimension $0$, ceci ramène 
à prouver \ref{IV:3-8} pour $\bar U$. Ce cas est justiciable de \ref{IV:3-7}. 
Le fait que $d f=0$ garantit que les points fixes sont tous de 
multiplicité un. 





\subsection{Remarque}\label{IV:3-9}

Pour $X$ de dimension $\leqslant 1$ sur $k$ algébriquement clos, et 
$f:X\to X$ tel que $d f=0$, il n'est pas difficile de vérifier que 
\ref{IV:3-8} est encore valable. 


