% !TEX root = sga4.5.tex

\chapter{Catégories dérivées}\label{VIII}










\section{Catégories triangulées: définitions et exemples}\label{VIII:1}





\subsection{Définition des catégories triangulées}\label{VIII:1-1}


\addtocounter{subsubsection}{-1}
\subsubsection{}\label{VIII:1-1-0}

Soient $\cA$ un catégorie additive, $T$ un automorphisme additif de $\cA$. 
$X$ et $Y$ étant deux objets de $\cA$, nous poserons 
\[
  \hom^i(X,Y) = \hom_\cA(X,T^i(Y)) \qquad i\in \dZ 
\]
Soient $X$, $Y$, $Z$ trois objets de $\cA$, $\alpha\in \hom^i(X,Y)$, 
$\beta\in \hom^j(X,Y)$. Nous définirons le composé $\beta\circ\alpha$ 
appartenant à $\hom^{i+j}(X,Z)$ par: 
\[
  \beta\circ \alpha = \beta\circ T^i(\alpha) \text{.}
\]
On vérifie qu'on obtient ainsi une nouvelle catégorie dont les groupes de 
morphismes entre les objets sont gradués. Cette nouvelle catégorie sera 
appelée, par abus de langage: la catégorie $\cA$, graduée par le foncteur 
de translation $T$. 

Soit $\cA$ une catégorie graduée par un foncteur de translation $T$. 
Lorsqu'on nommera ou notera un morphisme sans spécifier le degré, il 
s'agira toujours d'un morphisme de degré zéro. 

Soient $\cA$ et $\cA'$ deux catégories additives graduées par les foncteurs
$T$ et $T'$, $F$ un foncteur additif de $\cA$ dans $\cA'$. On dira que $F$ 
\emph{est gradué} si l'on s'est donné un isomorphisme de 
foncteurs\footnote{Texte modifié en 1976: le texte original demandait une 
égalité $F\circ T=T'\circ F$. Le cas des foncteurs à plusieurs variables 
est discuté dans \cite[XVIII 0.2]{sga4}. Il pose un problème de signes.} 
\[
  F\circ T = T'\circ F 
\]

La définition des morphismes de foncteurs gradués est donnée au 
\ref{VIII:4-1-1}. 

$\cA$ étant une categorie additive graduée par le foncteur $T$, on 
appellers triangle, un ensemble de trois objets $X$, $Y$, $Z$ et de trois 
morphismes $u:X\to Y$, $v:Y\to Z$, $w:Z\to T(X)$ ($\deg w=1$). Pour désigner 
les triangles on utilisera la notation: $(X,Y,Z,u,v,w)$ on bien le diagramme: 
\[\xymatrix{
  & Z \ar[dl]_-w \\
  X \ar[rr]^-u 
    & & Y \ar[lu]_-v 
}\qquad \deg(w)=1
\]
Soient $(X,Y,Z,u,v,w)$, $(X',Y',Z',u',v',w')$ deux triangles de $\cA$; un 
morphisme du premier triangle dans le second est un ensemble de trois 
morphismes $f:X\to X'$, $g:Y\to Y'$, $h:Z\to Z'$, tels que le diagramme 
suivant soit commutatif: 
\[\xymatrix{
  X \ar[r]^-u \ar[d]^-f 
    & Y \ar[r]^-v \ar[d]^-g 
    & Z \ar[r]^-w \ar[d]^-h 
    & T(X) \ar[d]^-{T(f)} \\
  X' \ar[r]^-{u'} 
    & Y' \ar[r]^-{v'} 
    & Z' \ar[r]^-{w'} 
    & T(X') 
}\]





\subsubsection{}\label{VIII:1-1-1}

On appelle \emph{catégorie triangulée} une catégorie additive graduée 
par un foncteur de translation, munie d'une famille de triangles qu'on appelle 
famille des triangles distingués. Cette famille doit vérifier de plus les 
axiomes: 


\paragraph*{TR1}
\phantomsection\label{VIII:TR1}

Tout triangle isomorphe à triangle distingué est distingué. Tout 
morphisme $u:X\to Y$ est contenu dans un triangle distingué 
$(X,Y,Z,u,v,w)$. Le triangle $(X,X,0,\operatorname{id}_{X'},0,0)$ est 
distingué. 


\paragraph*{TR2}
\phantomsection\label{VIII:TR2}

Pour que $(X,Y,Z,u,v,w)$ soit distingué, il faut et il suffit que \linebreak
$(Y,Z,T(X),v,w,-T(u))$ soit distingué. 


\paragraph*{TR3}
\phantomsection\label{VIII:TR3}

$(X,Y,Z,u,v,w)$, $(X',Y',Z',u',v',w')$ étant distingués, pour tout 
morphisme $(f,g):u\to u'$, il existe un morphisme $h:Z\to Z'$ tel que $(f,g,h)$ 
soit un morphisme de triangles. 


\paragraph*{TR4}
\phantomsection\label{VIII:TR4}

\[\xymatrix{
  & Z' \ar[dl]|{\deg=1} \\
  X \ar[rr]^-u 
    & & Y \ar[ul]_-i 
}\quad
\xymatrix{
  & X' \ar[dl]|{\deg(j)=1} \\
  Y \ar[rr]^-v 
    & & Z \ar[ul] 
}\quad
\xymatrix{
  & Y' \ar[dl]|{\deg=1} \\
  X \ar[rr]^-w 
    & & Z \ar[ul] 
}\]
étant trois triangles distingués tels que $w=v\circ u$, il existe deux 
morphismes $f:Z'\to Y'$, $g:Y' \to X'$ tels que 
\begin{enumerate}
  \item $(\operatorname{id}_X, v, f)$ soit un morphisme de triangle 
  \item $(u,\operatorname{id}_Z,g)$ soit un morphisme de triangle 
  \item $(Z',Y',X',f,g,T(i) j)$ soit un triangle distingué
\end{enumerate}

$\cC$ et $\cC'$ étant deux catégories triangulées, on appelle 
\emph{foncteur exact} de $\cC$ dans $\cC'$, tout foncteur additif, gradué, 
transformant les triangles distingués en triangles distingués. Les 
morphismes de foncteurs exacts sont les morphismes de foncteurs gradués. 

On déduit immédiatement dex axiomes \hyperref[VIII:TR1]{TR1}, 
\hyperref[VIII:TR2]{TR2}, \hyperref[VIII:TR3]{TR3} la propriété suivante: 





\begin{proposition}\label{VIII:1-1-2}
Soient $(X,Y,Z,u,v,w)$ un triangle distingué et $M$ un objet d'une 
catégorie triangulée. On a alors les suites exactes: 
\[\xymatrix{
  \cdots \ar[r] 
    & \hom^i(M,X) \ar[r]^-{\hom(M,u)} 
    & \hom^i(M,Y) \ar[r]^-{\hom(M,v)} 
    & \hom^i(M,Z) \ar[r]^-{\hom(M,w)} 
    & \hom^{i+1}(M,X) \ar[r] 
    & \cdots
}\]
\[\xymatrix{
  \cdots \ar[r] 
    & \hom^i(Z,M) \ar[r]^-{\hom(v,M)} 
    & \hom^i(Y,M) \ar[r]^-{\hom(u,M)} 
    & \hom^i(X,M) \ar[r]^-{\hom(w,M)} 
    & \hom^{i+1}(Z,M) \ar[r] 
    & \cdots
}\]
\end{proposition}

On déduit alors de cette proposition des énoncés du genre: 
\begin{itemize}
  \item $(f,g,h)$ étant un morphisme de triangle distingué et $f,g$ des 
    isomorphismes. 
  \item Les triangles distingués construits sur un morphisme 
    (\hyperref[VIII:TR1]{TR1}), sont tour isomorphismes (\emph{mais les 
    isomorphismes ne sont pas, en général, uniquoment déterminés}). 
  \item Lorsque dans un triangle distingué, un des morphismes admet un noyau 
    ou un conoyau, celui-ci se scinde i.e. il est de la forme: 
    \[\xymatrix{
      & Z+T(X) \ar[dl] \\
      X+Y \ar[rr] 
        & & Y+Z \ar[ul]
    }\]
  \item La somme de deux triangles distingués est distinguée. (On utilise 
    l'axiome \hyperref[VIII:TR4]{TR4}). 
\end{itemize}










\subsection{Exemples}\label{VIII:1-2}





\subsubsection{}\label{VIII:1-2-1}

Nous allons d'abord fixer quelques notations. Soit $\cA$ une catégorie 
additive $\eC(\cA)$ désignera la catégorie suivante: 
\begin{itemize}
  \item Les objets de $\eC(\cA)$ seront les complexes de $\cA$, sans limitation 
    de degré, à différentielle de degré $+1$. 
  \item Les morphismes de $\eC(\cA)$ seront les morphismes de complexes (qui 
    commutent avec la différentielle) conservant le degré. 
\end{itemize}

Soit $T$ le foncteur suivant: pour tout objet $X^\bullet$ de $\eC(\cA)$ 
\begin{align*}
  T(X^\bullet)_i &= (X^\bullet)_{i+1} \\
  d_i(T(X^\bullet)) &= -d_{i+1}(X^\bullet)  \text{.}
\end{align*}
Sur les morphismes le foncteur $T$ agit de la manière suivante: pour tout 
morphisme $f:X^\bullet \to Y^\bullet$, le morphisme 
$T(f):T(X^\bullet) \to T(Y^\bullet)$ est défini par 
\[
  T(f)_i = f_{i+1} \text{.}
\]
On vérifie immédiatement que $T$ est un automorphisme additif de 
$\eC(\cA)$. Les morphismes de degré $n$, définis à l'aide de ce foncteur 
de translation, sont alors les morphismes qui augmentent le degré de $n$, et 
qui commutent ou anticommutent avec la différentielle suivant la parité de 
$n$. On retrouve ainsi la définition généralement adoptée. 

$\sC$ désignera la famille de triangles suivante: un triangle 
$(X^\bullet,Y^\bullet,Z^\bullet,u,v,w)$ est un élément de la famille 
si\footnote{Texte modifié en 1976, pour assurer la validité de 
\ref{VIII:1-2-4}.}
\begin{itemize}
  \item $Z^\bullet$ est le complexe simple associé au complexe double 
    \[\xymatrix{
      \cdots \ar[r] 
        & 0 \ar[r] 
        & X^\bullet \ar[r]^-u 
        & Y^\bullet \ar[r] 
        & 0 \ar[r] 
        & \cdots 
    }\]
    o'ù les objets de $Y^\bullet$ sont les objets de premier degré zéro, 
    et les objets de $X^\bullet$, les objets de premier degré $-1$. 
  \item $v$ est l'image par le foncteur: complexe double $\mapsto$ complexe 
    simple, du morphisme de doubles complexes: 
    \[\xymatrix{
      \cdots \ar[r] 
        & 0 \ar[r] \ar[d] 
        & 0^\bullet \ar[r] \ar[d] 
        & Y^\bullet \ar[r] \ar[d] 
        & 0 \ar[r] \ar[d] 
        & \cdots \\
      \cdots \ar[r] 
        & 0 \ar[r] 
        & X^\bullet \ar[r] 
        & Y^\bullet \ar[r] 
        & 0 \ar[r] 
        & \cdots 
    }\]
  \item $w$ est l'opposé de l'image, par le même foncteur, du morphisme de 
    doubles complexes: 
    \[\xymatrix{
      \cdots \ar[r] 
        & 0 \ar[r] 
        & X^\bullet \ar[r] 
        & 0 \ar[r] 
        & 0 \ar[r] 
        & \cdots \\
      \cdots  \ar[r] 
        & 0 \ar[r] \ar[u] 
        & X^\bullet \ar[r] \ar[u] 
        & Y^\bullet \ar[r] \ar[u] 
        & 0 \ar[r] \ar[u] 
        & \cdots 
    }\]
\end{itemize}





\subsubsection{}\label{VIII:1-2-2}

L'ensemble des morphismes homotopes à zéro est un idéal bilatère (si 
$f$ et $g$ sont composables et si $f$ ou $g$ homotope à zéro, la composé 
est homotope à zéro. Si $f$ et $g$ sont homotopes à zéro, le morphisme 
$f\oplus g$ est homotope à zéro). Pour tout couple d'objet $X^\bullet$, 
$Y^\bullet$ on désigne par $\operatorname{Htp}(X^\bullet,Y^\bullet)$ le 
sous-groupe de $\hom_{\eC(\cA)}(X^\bullet,Y^\bullet)$ des morphismes homotopes 
à zéro. $\eK(\cA)$ sera alors la catégorie dont les objets sont les 
objets de $\eC(\cA)$ et les morphismes de source $X^\bullet$ et de but 
$Y^\bullet$, le groupe: 
\[
  \hom_{\eC(\cA)}(X^\bullet,Y^\bullet) / \operatorname{Htp}(X^\bullet,Y^\bullet) \text{.}
\]
La loi de composition sur les morphismes de $\eK(\cA)$ se définit à partir 
de la loi de composition sur les morphismes de $\eC(\cA)$ en passant au 
quotient. $\eK(\cA)$ est une catégorie additive. Le foncteur de translation 
$T$ sur $\eC(\cA)$ passe au quotient (le translaté d'un morphisme homotope 
à zéro, est homotope à zéro). Le foncteur obtenu est un automorphisme 
additif de $\eK(\cA)$ que nous noterons encore $T$. Par abus de notation 
$\eK(\cA)$ désignera par la suite la catégorie $\eK(\cA)$ graduée par le 
foncteur $T$. 

On appellera triangle distingué dans $\eK(\cA)$, tout triangle 
\emph{isomorphe} à un triangle provenant de $\sC$. La famille des triangles 
distingués vérifie les axioms \hyperref[VIII:TR1]{TR1}, 
\hyperref[VIII:TR2]{TR2}, \hyperref[VIII:TR3]{TR3} et 
\hyperref[VIII:TR4]{TR4}. Par abus de notation $\eK(\cA)$ désignera la 
catégorie $\eK(\cA)$ triangulée par la famille de triangles définie 
ci-dessus. 





\subsubsection{}\label{VIII:1-2-3}

On définit de plus tout un arsenal de sous-catégories: 
\begin{itemize}
  \item Un complexe est dit borné inférieurement si tous ses objets de 
    degré négatif, sauf au plus un nombre fini, sont nuls. On définit de 
    même les complexes bornés supérieurement, les complexes bornés. On 
    désignera par $\eK^+(\cA)$, $\eK^-(\cA)$, $\eK^b(\cA)$ les 
    sous-catégories pleines de $\eK(\cA)$ engendrées par ses familles 
    d'objets. Ces sous-catégories sont des ``\emph{sous-catégories 
    triangulées}'' de $\eK(\cA)$: tout triangle distingué dont deux des 
    objets sont des objets de la sous-catégorie, est isomorphe à un 
    triangle dont les trois objets sont des objets de la sous-catégorie. 
  \item Supposons que $\cA$ soit une catégorie abélienne. Un complexe est 
    dit cohomologiquement borné inférieurement si tous ses objets de 
    cohomologie de degré négatif sont nuls sauf un nombre fini d'entre 
    eux. On définit de manière analogue les complexes cohomologiquement 
    bornés supérieurement, les complexes cohomologiquement bornés, les 
    complexes acycliques. $\eK^{\infty,+}(\cA)$, $\eK^{\infty,-}(\cA)$, 
    $\eK^{\infty,b}(\cA)$, $\eK^{\infty,\varnothing}(\cA)$ désigneront les 
    sous-catégories pleines engendrées par ces familles d'objets. Ce sont 
    des sous-catégories triangulées. 
    
    On paiera aussi des sous-catégories triangulées $\eK^{+,b}(\cA)$, 
    $\eK^{+,\varnothing}(\cA)$, \ldots etc \ldots (le premier signe en 
    exposant donne des renseignements sur les objets des complexes, le 
    deuxième signe sur les objets de cohomologie). 
  \item $\cA$ n'étant plus nécessairement abélienne, soit $O$ un 
    ensemble d'objets de $\cA$ stable par isomorphisme et par somme directe. 
    $O(\cA)$ désignera la sous-catégorie pleine de $\eC(\cA)$ engendrée 
    par les complexes dont les objets en tout degré sont des objets de $O$. 
    On notera $\underline O(\cA)$ la sous-catégorie triangulée 
    correspondante dans $\eK(\cA)$. $\cA$ étant abélienne, on pourra 
    prendre pour ensemble $O$, l'ensemble $I$ des injectifs ou l'ensemble $P$ 
    des projectifs. 
\end{itemize}





\subsubsection{}\label{VIII:1-2-4}

$\eK(\cA)$ est uniquement déterminée par la donnée de $\eC(\cA)$ et de la 
famille $\sC$. De manière précise: tout foncteur additif gradué de 
$\eC(\cA)$ dans une catégorie triangulée, transformant tout triangle de 
$\sC$ en un triangle distingué, se factorise de manière unique par 
$\eK(\cA)$, et le foncteur obtenu est exact. 

Pour tout complexe $X^\bullet$ de $\eC(\cA)$, $\bar X^\bullet$ désignera le 
complexe obtenu à partir de $X^\bullet$ en annulant la différentielle. Une 
suite à trois termes: 
\[\xymatrix{
  0 \ar[r] 
    & X^\bullet \ar[r] 
    & Y^\bullet \ar[r] 
    & Z^\bullet \ar[r] 
    & 0 
}\]
sera dite suite semi-scindée si la suite 
\[\xymatrix{
  0 \ar[r] 
    & \bar X^\bullet \ar[r] 
    & \bar Y^\bullet \ar[r] 
    & \bar Z^\bullet \ar[r] 
    & 0 
}\]
est scindée; i.e. si elle est isomorphe à la suite: 
\[\xymatrix{
  0 \ar[r] 
    & \bar X^\bullet \ar[r] 
    & \bar X^\bullet + \bar Z^\bullet \ar[r] 
    & \bar Z^\bullet \ar[r] 
    & 0 \text{.}
}\]

Soit $0 \to X^\bullet \to Y^\bullet \to Z^\bullet \to 0$ une suite 
semi-scindée. Choisissons un scindage i.e. pour tout $n$ un isomorphisme 
\[
  (Y^\bullet)^n \simeq (X^\bullet)^n + (Z^\bullet)^n \text{.}
\]
La matrice de la différentielle de $Y^\bullet$, dans la décomposition en 
somme directe si-dessus est de la forme: 
\[
  \begin{pmatrix}
    d_{X^\bullet} & \varphi_n \\
    0 & d_{Y^\bullet} 
  \end{pmatrix}
\]
où $\varphi_n$ est un morphisme de $(Z^\bullet)^n$ dans 
$(X^\bullet)^{n+1}$. Les $\varphi_n$ déterminent un morphisme de complexes 
\[
  \varphi:Z^\bullet \to T(X^\bullet) \text{.}
\]
Lorsqu'on change le scindage, la classe modulo homotopie de $\varphi$ ne 
change pas. De plus, en notant $\underline u$, $\underline v$, 
$\underline\varphi$ les images dans $\eK(\cA)$ des morphismes $u$, $v$, 
$\varphi$, le triangle 
$(X^\bullet,Y^\bullet,Z^\bullet,\underline u,\underline v,\underline\varphi)$ 
est distingué. Désignons par $\mathsf{SSS}(\cA)$ la catégorie des 
suites semi-scindées de $\eC(\cA)$, et par $\mathsf{TrK}(\cA)$ la 
catégorie des triangles distingués de $\eK(\cA)$. Ce qui précède 
permet de définir un foncteur: 
\[
  \rho:\mathsf{SSS}(\cA) \to \mathsf{TrK}(\cA) \text{.}
\]
Ce foncteur est essentiellement surjectif. 










\subsection{Exemples de foncteurs cohomologiques et de foncteurs exacts}\label{VIII:1-3}





\begin{definition}
Soient $\cA$ une catégorie triangulée, $\cB$ une catégorie abélienne. 
Un foncteur additif $F:\cA\to \cB$ est die \emph{foncteur cohomologique} si 
pour tout triangle distingué $(X,Y,Z,u,v,w)$ la suite 
\[\xymatrix{
  F(X) \ar[r]^-{F(u)} 
    & F(Y) \ar[r]^-{F(v)} 
    & F(Z) 
}\]
est exacte. 
\end{definition}

Le foncteur $F\circ T^i$ sera souvent noté $F^i$. En vertu de l'axiome 
\hyperref[VIII:TR2]{TR2} des catégories triangulées, on a la suite 
exacte illimitée: 
\[\xymatrix{
  \cdots \ar[r] 
    & F^i(X) \ar[r] 
    & F^i(Y) \ar[r] 
    & F^i(Z) \ar[r] 
    & F^{i+-}(X) \ar[r] 
    & \cdots 
}\]





\subsubsection{Exemples}\label{VIII:1-3-2}

\begin{enumerate}
  \item Soit $\cA$ une catégorie triangulée. Pour tout objet $X$ de $\cA$ 
    le foncteur $\hom_\cA(X,-)$ est un foncteur cohomologique. Le foncteur 
    $\hom_\cA(-,X)$ est un foncteur cohomologique $\cA^\circ\to\mathsf{Ab}$. 
  \item Soient $\cA$ une catégorie abélienne, $\eC(\cA)$ la catégorie des 
    complexes de $\cA$. On désigne par $\h^0:\eC(\cA)\to\cA$ le foncteur 
    suivant: Soit $X^\bullet$ un objet de $\eC(\cA)$. On pose 
    \[
      \h^0(X^\bullet) = \ker d^0 / \im{d^{-1}} \text{.}
    \]
    $\h^0$ annule les morphismes homotopes à zéro, donc se factorise de 
    manière unique par $\eK(\cA)$. On désignera encore par 
    $\h^0:\eK(\cA) \to \cA$ le foncteur obtenu. 
\end{enumerate}





\subsubsection{Exemple de bi-foncteur exact}\label{VIII:1-3-3}

Soient $\cA$, $\cA'$, $\cA''$ trois catégories additives, 
\[
  P:\cA\times \cA'\to \cA''
\]
un foncteur bilinéaire (i.e. additif par rapport à chacun des arguments 
séparément). On en déduit alors le foncteur bilinéaire: 
\[
  P^\ast:\eC(\cA)\times \eC(\cA') \to \eC(\cA'') 
\]
de la manière suivante: 

Soient $X^\bullet$ un objet de $\eC(\cA)$ et $Y^\bullet$ un objet de 
$\eC(\cA')$. $P(X^\bullet,Y^\bullet)$ est un complexe double de $\cA''$. On 
pose alors: $P^\ast(X^\bullet,Y^\bullet) = $ complexe simple associé à 
$P(X^\bullet,Y^\bullet)$. 

Soient $f$ un morphisme de $\eC(\cA)$ (resp. $\eC(\cA')$) homotope à zéro 
et $Z^\bullet$ un objet de $\eC(\cA')$ (resp. $\eC(\cA)$). Le morphisme 
$P^\ast(f,Z^\bullet)$ (resp. $P^\bullet(Z^\bullet,f)$) est alors homotope à 
zéro. On en déduit que $P^\ast$ définit d'une manière unique un 
foncteur: 
\[
  P^\bullet:\eK(\cA) \times \eK(\cA') \to \eK(\cA'') \text{.}
\]
$P^\bullet$ est un bi-foncteur exact. 

En particulier, soit $\cA$ un catégorie additive. On peut prendre pour $P$ le 
foncteur: 
\begin{align*}
  \cA^\circ\times \cA &\to \mathsf{Ab} \\
  (X,Y) &\mapsto \hom(X,Y) 
\end{align*}

On obtient alors par la construction précédente un foncteur 
\[
  \hom^\bullet:\eK(\cA)^\circ \times \eK(\cA) \to \eK(\mathsf{Ab}) 
\]
qui, composé avec le foncteur $\h^0:\eK(\mathsf{Ab})\to \mathsf{Ab}$, 
redonne évidemment le foncteur $\hom_{\eK(\cA)}$. 




















\section{Les catégories quotients}\label{VIII:2}










\subsection{Sous-catégories épaisses. Systèmes multiplicatifs de morphismes}\label{VIII:2-1}

Nous commencerons par donner deux définitions. 





\begin{definition}\label{VIII:2-1-1}
Une sous-catégorie $\cB$ d'une catégorie triangulée $\cA$ est dite 
\emph{épaisse} si $\cB$ est une sous-catégorie triangulée pleine de $\cA$ 
et si de plus $\cB$ possède la propriété suivante: 

Pour tout morphisme $f:X\to Y$, se factorisant par un objet de $\cB$ et contenu 
dans un triangle distingué $(X,Y,Z,f,g,h)$, où $Z$ est un objet de $\cB$, 
la source de $f$ et le but de $f$ sont des objets de $\cB$. 
\end{definition}

L'ensemble des sous-catégories épaisses de $\cA$ sera noté $\sN$. 

L'intersection d'une famille quelconque de sous-catégories épaisses est 
une sous-catégorie épaisse. 





\begin{definition}
Soit $\cA$ un catégorie triangulée. Un ensemble $S$ de morphismes de $\cA$ 
est appelé \emph{système multiplicatif de morphismes} si'il posède les 
cinq propriétés suivantes: 

\paragraph*{FR1}
\phantomsection\label{VIII:FR1}
Si $f,g\in S$ et si $f$ et $g$ sont composables, $f\circ g\in S$. Pout tout 
objet $X$ de $\cA$, le morphisme identique de $X$ est un élément de $S$. 

\paragraph*{FR2}
\phantomsection\label{VIII:FR2}
Dans la catégorie $\cA$, tout diagramme 
\[\xymatrix{
  & Y \ar[d]^-{s\in S} \\
  Z \ar[r]^-f 
    & X 
}\]
se complète en un diagramme commutatif 
\[\xymatrix{
  P \ar[r]^-g \ar[d]^-{t\in S} 
    & Y \ar[d]^-{s\in S} \\
  Z \ar[r]^-f 
    & X 
}\]
De plus la propriété symétrique est vraie. 

\paragraph*{FR3}
\phantomsection\label{VIII:FR3}
Si $f$ et $g$ sont des morphismes, les propriétés suivantes sont 
équivalentes: 
\begin{enumerate}[\indent i)]
  \item il existe un $s\in S$ tel que $s\circ f=s\circ g$ 
  \item il existe un $t\in S$ tel que $f\circ t=g\circ t$
\end{enumerate}

\paragraph*{FR4}
\phantomsection\label{VIII:FR4}
Soit $T$ le foncteur de translation de $\cA$. Par tout élément $s$ de $S$, 
$T(s)\in S$. 

\paragraph*{FR5}
\phantomsection\label{VIII:FR5}
$(X,Y,Z,u,v,w)$, $(X',Y',Z',u',v',w')$ étant deux triangles distinguées, 
$f$ et $g$ deux éléments de $S$ tels que le couple $(f,g)$ soit un 
morphisme de $u$ dans $u'$, il existe un morphisme $h\in S$, tel que 
$(f,g,h)$ soit un morphisme de triangle. 
\end{definition}

Un système multiplicatif de morphismes est dit \emph{saturé} s'il possède 
la propriété suivante: 

Un morphisme $f$ appartient à $S$ si et seulement s'il existe deux 
morphismes $g$ et $g'$ tels que $g\circ f\in S$ et $f\circ g'\in S$. 

L'intersection d'une famille quelconque de systèmes multiplicatifs saturés 
un système multiplicatif saturé. 

L'ensemble des systèmes multiplicatifs saturés sera noté $\sS$. 

Soit $\cB$ une sous-catégorie épaisse; on désigne par $\varphi(\cB)$ 
l'ensemble des morphismes $f$ qui sont contenus dans un triangle distingué 
$(X,Y,Z,f,g,h)$ où $Z$ est un objet de $\cB$. $\varphi(\cB)$ est un 
système multiplicatif saturé. 

Soit $S$ un système multiplicatif saturé; on désigne par $\psi(S)$ la 
sous-catégorie pleine engendrée par les objets $Z$ contenus dans un 
triangle distingué $(X,Y,Z,f,g,h)$ où $f$ est un élément de $S$. La 
sous-catégorie $\psi(S)$ est une sous-catégorie épaisse. 

Le résultat principal de ce numéro est alors le suivant: $\varphi$ est un 
isomorphisme (conservant la relation d'ordre définie par l'inclusion) de 
$\sN$ sur $\sS$. L'isomorphisme inverse n'est autre que $\psi$. 










\subsection{Problèmes universels}\label{VIII:2-2}

Soient $\cA$ et $\cA'$ deux catégories triangulées et $F$ un foncteur 
exacte de $\cA$ dans $\cA'$. Soient $S(F)$ l'ensemble des morphismes de $\cA$ 
qui sont transformés par $F$ en isomorphismes et $\cB(F)$ la sous-catégorie 
pleine engendrée par les objets de $\cA$ qui sont transformés par $F$ en 
objets nuls de $\cA'$. $S(F)$ est un système multiplicatif saturé et 
$\cB(F)$ est une sous-catégorie épaisse. De plus $S(F) = \varphi(\cB(F))$. 
Soient alors $\cA$ une catégorie triangulée, $\cB$ une sous-catégorie 
épaisse et $S=\varphi(\cB)$ le système multiplicatif saturé 
correspondant. Les deux problèmes universels ci-dessous sont équivalents. 


\paragraph*{Problème 1}
\phantomsection\label{VIII:prob1}
Trouver une catégorie triangulée $\cA/\cB$ et un foncteur exact 
$Q:\cA\to \cA/\cB$ tels que tout foncteur exact de $\cA$ dans une catégorie 
triangulée $\cA'$, transformant les objets de $\cB$ en objets nuls de 
$\cA'$, admette, de manière unique, une factorisation de la forme $G\circ Q$ 
où $G$ est un foncteur exact. 


\paragraph*{Problème 2}
\phantomsection\label{VIII:prob2}
Trouver une catégorie triangulée $\cA_S$ et un foncteur exact 
$Q:\cA\to \cA_S$ tels que tout foncteur exact de $\cA$ dans une catégorie 
triangulée $\cA'$, transformant les morphismes de $S$ en isomorphismes de 
$\cA'$, admette, de manière unique, une factorisation de la forme 
$G\circ Q$ où $G$ est un foncteur exact. 

(Le rédacteur prie le lecteur de bien vouloir l'excuser pour l'emploi de ce 
langage désuet et rétrograde.) Nous allons montrer, dans le numéro 
suivant, que le \hyperref[VIII:prob2]{problème 2} admet une solution. Donc 
le \hyperref[VIII:prob1]{problème 1} correspondant admet la même solution. 










\subsection{Calcul de fraction}\label{VIII:2-3}





\subsubsection{}\label{VIII:2-3-1}

Soient $\cA$ un catégorie triangulée, $S$ un système multiplicatif 
saturé. Désignons par $\cA[S^{-1}]$ la catégorie de fraction de $\cA$ 
pour l'ensemble de morphismes $S$ et $Q$ le foncteur canonique de $\cA$ dans 
$\cA[S^{-1}]$. Le couple $(\cA[S^{-1}],Q)$ résout le problème suivant: Tout 
foncteur de $\cA$ dans une catégorie quelconque (non nécessairement 
additive) transformant morphisme de $S$ en isomorphisme, se factorise, d'une 
manière unique, par $(\cA[S^{-1}],Q)$. Un tel couple existe toujours, sans 
hypothèse sur l'ensemble de morphismes $S$ \cite[1.1]{gz67}\footnote{The 
original text references [C.G.G] here. Most likely, this is the lost work 
\emph{Catégories et foncteurs} by Chevalley, Gabriel and Grothendieck, 
referenced in\cite{sc72}. }. Cependant l'ensemble 
de morphismes $S$, étant un syst\`me multiplicatif saturé, possède les 
propriétés \hyperref[VIII:FR1]{FR1}, \hyperref[VIII:FR2]{FR2}, 
\hyperref[VIII:FR3]{FR3}. Par suite, d'après \cite{gz67}, la catégorie 
$\cA[S^{-1}]$ peut s'obtenir par un calcul de fractions, à droite ou à 
gauche. Nous allons rappeler comment on obtient $\cA[S^{-1}]$ par un calcul de 
fractions à droite. (Le calcul de fractions à gauche s'obtient par le 
procédé du renversement des flèches.) 





\subsubsection{}\label{VIII:2-3-2}

Les objets de $\cA[S^{-1}]$ sont le objets de $\cA$. 
\begin{itemize}
  \item Pour tout objet $X$ de $\cA$, désignons par $S_X$ la catégorie des 
    flèches appartenant à $S$, ayant pour but $X$. A tout objet $Y$ de 
    $\cA$, on associe, fonctoriellement en $Y$, le foncteur suivant sur 
    $S_X^\circ$ à valeur dans la catégorie des ensembles: 
    ($S_X^\circ$ désigne la catégorie opposée) 
    \[
      H_Y:s\mapsto \hom(\operatorname{source}(s),Y) \text{.}
    \]
    On pose alors: 
    \[
      \hom_{\cA[S^{-1}]}(X,Y) = \varinjlim_{S_X^\circ} H_Y \text{.}
    \]
    Pour tout ce qui concerne les limites inductives, on se référera à 
    \cite{ar62}. 
    
    On constatera alors, avec plaisir en utilisant \hyperref[VIII:FR1]{FR1}, 
    \hyperref[VIII:FR2]{FR2}, \hyperref[VIII:FR3]{FR3} que la catégorie 
    $S_X^\circ$ possède les propriétés L1, L2, L3. Les limites inductives 
    possèdent donc toutes les bonnes propriétés des limites inductives 
    sur les ensembles ordonnés filtrants. 
  \item Soient $X$, $Y$, $Z$, trois objets de $\cA$, $a$ un élément de 
    $\hom_{\cA[S^{-1}]}(X,Y)$ et $b$ un élément de 
    $\hom_{\cA[S^{-1}]}(Y,Z)$. Soient $s$ un objet de $S_X$, $\underline a$ un 
    élément de $\hom(\operatorname{source}(s),Y)$ dont l'image est $a$ et 
    un objet de $S_y$, $\underline b$ un élément de 
    $\hom(\operatorname{source}(t),Z)$ dont l'image est $b$. On a alors un 
    diagramme: 
    \begin{equation*}\tag{1}\label{VIII:eq:2-3-2-1}
    \xymatrix{
      & \bullet \ar[dl]_-s \ar[dr]^-{\underline a} 
        & 
        & \bullet \ar[dl]_-t \ar[dr]^-{\underline b} \\
      X 
        & 
        & Y 
        & 
        & Z 
    }
    \end{equation*}
    que, d'après \hyperref[VIII:FR2]{FR2}, on peut compléter en un 
    diagramme: 
    \[\xymatrix{
      & & \bullet \ar[dl]_-{t'} \ar[dr]^-{\underline c} \\
      & \bullet \ar[dl]_-s \ar[dr]^-{\underline a} 
        & & \bullet \ar[dl]_-t \ar[dr]^-{\underline b} \\
      X 
        & & Y 
        & & Z 
    }\]
    Notons $b\circ a$ l'image dans $\hom_{\cA[S^{-1}]}(X,Z)$ du morphisme 
    $\underline{b\circ c}$. On vérifie que gr\^ace aux propriétés 
    \hyperref[VIII:FR1]{FR1}, \hyperref[VIII:FR2]{FR2}, 
    \hyperref[VIII:FR3]{FR3}, $b\circ a$ ne dépend pas des représentants 
    $\underline a$ et $\underline b$ choisis et qu'il ne dépend pas non plus 
    de la manière de compléter le diagramme \eqref{VIII:eq:2-3-2-1}. On 
    vérifie de plus qu'on a ainsi défini une catégorie $\cA[S^{-1}]$. 
\end{itemize}

Nous noterons $Q$ le foncteur évident: $\cA\to \cA[S^{-1}]$. 

$\cA$ étant une catégorie additive, $\cA[S^{-1}]$ est une catégorie 
additive. De plus, on démontre en utilisant \hyperref[VIII:FR4]{FR4} qu'il 
existe un et un seul foncteur de translation sur $\cA[S^{-1}]$, que nous 
noterons $T$, vérifiant la relation: 
\[
  Q\circ T = T\circ Q \text{.}
\]
Enfin, à l'aide de \hyperref[VIII:FR5]{FR5}, on démontre qu'il existe une 
et une seule structure triangulée sur $\cA[S^{-1}]$ telle que le foncteur $Q$ 
soit exact. Les triangles distingués de $\cA[S^{-1}]$ ne sont qutre que les 
triangles isomorphes aux images, par $Q$, des triangles distingués de $\cA$. 
En désignant par $\cA_S$, la catégorie $\cA[S^{-1}]$ muni de cette 
structure triangulée, on démontre sans difficulté que le couple 
$(\cA_S,Q)$ est une solution au \hyperref[VIII:prob2]{problème 2}. 





\begin{definition}\label{VIII:2-3-3}
Soient $\cA$ un catégorie triangulée, $\cB$ une sous-catégorie épaisse 
de $\cA$, $(Q,\cA/\cB)$ la solution du \hyperref[VIII:prob1]{problème 1} 
(\S \ref{VIII:2-2}). $\cA/\cB$ sera appelée la \emph{catégorie quotient} de 
$\cA$ par $\cB$. $Q$ sera appelé le foncteur canonique de passage au 
quotient. Plus généralement soit $\cN$ une sous-catégorie triangulée de 
$\cA$ et soit $\underline\cN$ la plus petit sous-catégorie épaisse 
contenant $\cN$. La catégorie quotient de $\cA$ par $\cN$: $\cA/\cN$ sera par 
définition $\cA/\underline\cN$. 
\end{definition}










\subsection{Propriétés des catégories quotients}\label{VIII:2-4}





\subsubsection{}\label{VIII:2-4-1}

Soient $\cA$ une catégorie triangulée et $H$ un foncteur cohomologique à 
valeur dans une catégorie abélienne $\cG$. Soient $\cB_H$ la 
sous-catégorie pleine engendrée par les objets dont tous les translatés 
sont transformés par $H$ en objets nuls de $\cG$ et $S_H$ l'ensemble des 
morphismes dont tous les translatés sont transformés par $H$ en 
isomorphismes de $\cG$. $\cB_H$ est une sous-catégorie épaisse de $\cA$; 
$S_H$ est un système multiplicatif saturé. De plus $S_H=\varphi(\cB_H)$ 
(\S \ref{VIII:2-1}). Le foncteur $H$ se factorise d'un manière unique par 
$(Q,\cA/\cB_H)$. 





\begin{theorem}\label{VIII:2-4-2}
Soient $\cA$ une catégorie triangulée, $\cB$ un sous-catégorie 
triangulée, $\cN$ une sous-catégorie épaisse de $\cA$, $S=\varphi(\cN)$ 
le système multiplicatif saturé correspondant. 
\begin{enumerate}[(a)]
  \item La catégorie $\cN\cap\cB$ est une sous-catégorie épaisse de 
    $\cB$. Le système multiplicatif correspondant est $S\cap \cB$. 
  \item Les deux propriétés suivantes sont équivalentes: 
    \begin{enumerate}[(i)]
      \item Pour tout objet $X$ de $\cB$ et tout morphisme $s:R\to X$ où $s$ 
        est un élément de $S$, il existe un morphisme $s':R'\to R$ où 
        $s\circ'$ est un élément de $S\cap \cB$; et 
        $R'\in\operatorname{Ob}\cB$. 
      \item Tout morphisme d'un objet $X$ de $\cB$ dans un objet $Y$ de $\cN$ 
        se factorise par un objet de $\cN\cap \cB$. 
    \end{enumerate}
  \item Les propriétés (i)' et (ii)' obtenues à partir de (i) et (ii) en 
    renversant les flèches sont équivalents. 
  \item Si les proprietés (i) et (ii) sont vérifiées ou bien si les 
    propriétés (i)' et (ii)' sont vérifiées, le foncteur canonique 
    \[\xymatrix{
      \cB/\cN\cap \cB \ar[r] & \cA/\cN
    }\]
    est fidèle. 
    
    Comme ce foncteur est injectif sur les objets, il réalise 
    $\cB/\cN\cap\cB$ comme sous-catégorie de $\cA/\cN$. 
  \item Si de plus $\cB$ est une sous-catégorie pleine de $\cA$, 
    $\cB/\cN\cap \cB$ est une sous-catégorie pleine de $\cA/\cN$. 
\end{enumerate}
\end{theorem}





\begin{corollary}\label{VIII:2-4-3}
Soient $\cA\subset \cB\subset \cC$ trois catégories triangulées, $\cA$ sous 
catégorie épaisse de $\cB$, $\cB$ sous-catégorie épaisse de $\cC$. 
Alors $\cA$ est une sous-catégorie épaisse de $\cC$, $\cB/\cA$ est une 
sous-catégorie épaisse de $\cC/\cA$. Le foncteur canonique 
$\cC/\cB \to (\cC/\cA)/(\cB/\cA)$ est un isomorphisme. 
\end{corollary}










\subsection{Propriétés du foncteur de passage au quotient}\label{VIII:2-5}

Soient $\cA$ un catégorie triangulée, $\cB$ un sous-catégorie épaisse, 
$S$ le système multiplicatif correspondant, $Q:\cA\to \cA/\cB$ le foncteur 
canonique de passage au quotient. 





\begin{proposition}\label{VIII:2-5-1}
$X$ et $Y$ étant des objets de $\cA$, les assertions suivantes sont 
équivalentes: 
\begin{enumerate}[(a)]
  \item Tout morphisme $f:X\to Y$, tel qu'il existe un morphisme $s:Z\to X$, de 
    $S$, vérifiant $f\circ s=0$, est un morphisme nul.
  \item Tout morphisme $f:X\to Y$, tel qu'il existe un morphisme $s:Y\to Z$, 
    $s\in S$, vérifiant $s\circ f=0$, est un morphisme nul. 
  \item Tout morphisme $f:X\to Y$ qui se factorise par un objet de $\cB$ est 
    nul. 
  \item L'application canonique: 
    \[
      \hom_\cA(X,Y) \to \hom_{\cA/\cB}(Q(X),Q(Y))
    \]
    est injective. 
\end{enumerate}
\end{proposition}





\begin{proposition}\label{VIII:2-5-2}
Soient $X$ et $Y$ deux objets de $\cA$. Les assertions suivantes sont 
équivalentes: 
\begin{enumerate}[(a)]
  \item Tout diagramme ($s\in S$): 
    \[\xymatrix{
      & Z \ar[dl]_-s \ar[dr]^-f \\
      X 
        & & Y 
    }\]
    se compl`ete en un diagramme: $(s,t\in S$): 
    \[\xymatrix{
      & Z' \ar[d]_-t \\
      & Z \ar[dl]_-s \ar[dr]^-f \\
      X 
        & & Y 
    }\]
    où $f\circ t$ se factorise par $(X,s\circ t)$. 
  \item Assertion obtenue en renversant le sens de flèches dans (a) et en 
    permutant $X$ et $Y$. 
  \item Tout diagramme: 
    \[\xymatrix{
      X \ar[r]^-f 
        & N \ar[r]^-g 
        & Y 
    }\]
    où $g\circ f=0$ et où $N$ est un objet de $\cB$, se complète en un 
    diagramme: 
    \[\xymatrix{
      & N' \ar[d]^-i \\
      X \ar[ur]^-h \ar[r]^-f 
        & N \ar[r]^-g 
        & Y 
    }\]
    où $f=i\circ h$ et où $g\circ i=0$. 
  \item Assertion obtenue en changeant le sens des flèches dans (c) et en 
    permutant $X$ et $Y$. 
  \item L'application canonique: 
    \[\xymatrix{
      \hom_\cA(X,Y) \ar[r]^-Q 
        & \hom_{\cA/\cB}(Q(X),Q(Y)) 
    }\]
    est surjective. 
\end{enumerate}
\end{proposition}





\begin{proposition}\label{VIII:2-5-3}
Soit $X$ un objet de $\cA$. Les assertions suivantes sont équivalentes: 
\begin{enumerate}[(a)]
  \item Pour tout objet $Y$ de $\cA$, l'application canonique 
    \[\xymatrix{
      \hom_\cA(X,Y) \ar[r]^-Q 
        & \hom_{\cA/\cB}(Q(X),Q(Y)) 
    }\]
    est un isomorphisme. 
  \item Tout morphisme de $X$ dans un objet de $\cB$ est nul. 
\end{enumerate}
De même, les deux assertions suivantes sont équivalentes: 
\begin{enumerate}[(a)']
  \item Pour tout objet $Y$ de $\cA$, l'application canonique: 
    \[\xymatrix{
      \hom_\cA(X,Y) \ar[r]^-Q 
        & \hom_{\cA/\cB}(Q(X),Q(Y)) 
    }\]
    est un isomorphisme. 
  \item Tout morphisme d'un objet de $\cB$ dans $X$ est nul. 
\end{enumerate}
\end{proposition}





\begin{definition}\label{VIII:2-5-4}
Tout objet possédant les propriétés équivalentes (a) et (b) de la 
proposition précédente, sera appelé: \emph{objet libre à gauche sur 
$\cA/\cB$} ou bien encore objet \emph{$Q$-libre à gauche}. De même, tout 
objet possédant les propriétés équivalentes (a)' et (b)' sera appelé 
\emph{objet libre à droite sur $\cA/\cB$} ou encore objet 
\emph{$Q$-libre à droite}. Il est défini à isomorphisme près par la 
connaissance de $Q(X)\in\operatorname{Ob}{\cA/\cB}$. 
\end{definition}





\subsection{Foncteurs adjoints au foncteur de passage au quotient}\label{VIII:2-6}





\begin{definition}\label{VIII:2-6-1}
Deux sous-catégories triangulées $\cN$ et $\cN'$ d'une catégorie 
triangulée $\cA$ sont dites \emph{orthagonales} si pour tout objet $X$ de 
$\cN$ et tout objet $Y$ de $\cN'$, on a: 
\[
  \hom_\cA(X,Y) = 0 
\]
$\cN'$ sera dite alors orthogonale à droite à $\cN$ et $\cN$ orthogonale 
à gauche à $\cN'$. 
\end{definition}





\begin{proposition}\label{VIII:2-6-2}
Soit $\cN$ une sous-catégorie triangulée d'une catégorie triangulée 
$\cA$. La catégorie pleine $\cN^\bot$ (resp. $^\bot\cN$) engendrée par les 
objets $X$ de $\cA$ tels que pour tout objet $Y$ de $\cN$ on ait 
$\hom_\cA(Y,X)=0$ (resp. $\hom_\cA(X,Y)=0$), est une sous-catégorie épaisse 
de $\cA$. La catégorie $\cN^\bot$ est appelée par abus de langage, 
l'orthogonale à droite de $\cN$. 
\end{proposition}

Cette proposition nous permet de traduire la proposition \ref{VIII:2-5-3}. 





\begin{proposition}\label{VIII:2-6-3}
Soit $\cB\subset\cA$, une sous-catégorie épaisse d'une catégorie 
triangulée $\cA$. La catégorie pleine engendrée par les objets libre à 
droite sur $\cA/\cB$ (resp. à gauche), n'est autre que l'orthogonale à 
droite (resp. à gauche) de $\cB$. 
\end{proposition}





\begin{proposition}\label{VIII:2-6-4}
Soient $\cA$ une catégorie triangulée, $\cB$ une sous-catégorie 
épaisse de $\cA$, $\cB^\bot$ l'orthogonale à droite de $\cB$, $S(\cB)$, 
$s(\cB^\bot)$ les systèmes multiplicatifs correspondants, $Q$ et $Q^\bot$ les 
foncteurs canoniques de passage au quotient. Considérons pour un objet $x$ de 
$\cA$, les propriétés suivantes: 
\begin{enumerate}[(i)]
  \item $X$ s'envoie par un morphisme de $S(\cB)$ dans un objet de $\cB^\bot$. 
  \item $X$ reçoit par un morphisme de $S(\cB^\bot)$ un objet de $\cB$. 
  \item La catégorie $S_X(\cB)$ des flèches de $S(\cB)$ de source $X$, 
    admet un objet final. 
  \item La catégorie $S^X(\cB^\bot)$ des flèches de $S(\cB^\bot)$ de but 
    $X$, admet un objet initial. 
  \item La catégorie $\cB/X$ des objets de $\cB$ au-dessus de $X$, admet un 
    objet final. 
  \item La catégorie $\cB^\bot/X$ des objets de $\cB^\bot$ au-dessous de $X$ 
    admet un objet initial. 
\end{enumerate}
On a alors 
\[\xymatrix@=0.5cm{
  & & & & (iv) \\
  (i) \ar@{<=>}[r] 
    & (ii) \ar@{<=>}[r] 
    & (iii) \ar@{<=>}[r] 
    & (v) \ar@{=>}[ur] \ar@{=>}[dr] \\
  & & & & (vi) 
}\]

Si de plus $^\bot(\cB^\bot)=\cB$ alors toutes les propriétés sont 
équivalentes. 
\end{proposition}

Soient $\cB$ une sous-catégorie épaisse d'une catégorie triangulée 
$\cA$, le foncteur d'injection $i$ de $\cB$ dans $\cA$ et $Q$ le foncteur de 
passage au quotient de $\cA$ dans $\cA/\cB$, $\cB^\bot$ l'orthogonale à 
droite de $\cB$, les foncteurs correspondant $i^\bot$ et $Q^\bot$. On déduit 
immédiatement de la proposition \ref{VIII:2-6-4} les propositions suivantes: 





\begin{proposition}\label{VIII:2-6-5}
Les deux propriétés suivantes sont équivalentes: 
\begin{enumerate}[(i)]
  \item Le foncteur $i$ admet un adjoint à droite. 
  \item Le foncteur $Q$ admet un adjoint à droite. 
\end{enumerate}
\end{proposition}

La proposition est encore vraie lorsqu'on remplace le mot droite par le omt 
gauche. 





\begin{proposition}\label{VIII:2-6-6}
Les deux propriétés suivantes sont équivalentes: 
\begin{enumerate}[(i)]
  \item Le foncteur $i$ admet un adjoint à droite. 
  \item Le foncteur $i^\bot$ admet un adjoint à gauche. L'orthogonale à 
    gauche de la catégorie $\cB^\bot$ est égale à $\cB$. 
\end{enumerate}
\end{proposition}





\begin{proposition}\label{VIII:2-6-7}
Supposons vérifiées les propriétés des propositions \ref{VIII:2-6-5} et 
\ref{VIII:2-6-6}. Soient $i^\ast$ et $Q^\ast$ les adjoints à droite de $i$ et 
$Q$. De même soient $^\ast i^\bot$ et $^\ast Q^\bot$ les adjoints à gauche 
de $i^\bot$ et de $Q^\bot$. Tous ces foncteurs sont exactes. De plus: 

Il existe un isomorphisme fonctoriel 
$Q^\ast\circ Q\iso i^\bot\circ ^\ast i^\bot$ tel que le diagramme ci-après 
soit commutatif: 
\[\xymatrix{
  & \operatorname{id} \ar[dl] \ar[dr] \\
  Q^\ast \circ Q \ar[rr]^\sim 
    & & i^\bot\circ ^\ast i^\bot 
}\]
(les flèches obliques sont définies par les propriétés d'adjonction).

Il existe de même un isomorphisme 
$^\ast Q^\bot \circ Q^\bot \iso i\circ i^\ast$ tel que le diagramme suivant soit 
commutatif: 
\[\xymatrix{
  ^\ast Q^\bot\circ Q^\bot \ar[rr]^-\sim \ar[dr] 
    & & i\circ i^\ast \ar[dl] \\
  & \operatorname{id}{}
}\]

Enfin, il existe un morphisme fonctoriel de degré $1$: 
$\partial:i^\bot\circ ^\ast i^\bot \to i\circ i^\ast$ tel que le le triangle 
suivant soit distingué $(\deg\partial=1$): 
\[\xymatrix{
  i^\bot \circ ^\ast i^\bot \ar[rr]^-\partial 
    & & i\circ i^\ast \ar[dl] \\
  & \operatorname{id}{} \ar[ul] 
}\]

Un tel morphisme $\delta$ est unique et vérifie 
$\delta(T X_0) = - T \delta(X)$ ($X$ objet $q\circ q$, $T$ translation). 
\end{proposition}















\section{Les catégories dérivées d'une catégorie abélienne}\label{VIII:3}





\subsection{Les catégories dérivées}\label{VIII:3-1}

On utilise les notations du \ref{VIII:1-2-3}. On utilisera de plus les 
notations suivantes: 

\subsubsection{Notations}\label{VIII:3-1-1}

Soit $\cA$ une catégorie abélienne. On pose: 
\begin{align*}
  \eD(\cA) &= \eK^{\infty,\infty}(\cA)/\eK^{\infty,\varnothing}(\cA) \\
  \eD^b(\cA) &= \eK^{b,b}(\cA)/\eK^{b,\varnothing}(\cA) \\
  \eD^+(\cA) &= \eK^{+,+}(\cA) / \eK^{+,\varnothing}(\cA) \\
  \eD^-(\cA) &= \eK^{-,-}(\cA) / \eK^{-,\varnothing}(\cA) \text{.}
\end{align*}





\begin{proposition}\label{VIII:3-1-2}
Les foncteurs naturels qui figurent dans le diagramme suivant: 
\[\xymatrix@=0.5cm{
  & \eD(\cA) \\
  \eD^+(\cA) \ar[ur] 
    & & \eD^-(\cA) \ar[ul] \\
  & \eD^b(\cA) \ar[ul] \ar[ur] 
}\]
sont pleinement fidèles. 

Les foncteurs naturels: 
\begin{align*}
  \eD^b(\cA) &= \eK^{b,b}(\cA)/\eK^{b,\varnothing}(\cA) \to \eK^{+,b}(\cA)/\eK^{+,\varnothing}(\cA) \\
  \eD^b(\cA) &= \eK^{b,b}(\cA)/\eK^{b,\varnothing}(\cA) \to \eK^{-,b}(\cA) / \eK^{-,\varnothing}(\cA) 
\end{align*}
sont des équivalences de catégories. 

Les foncteurs du diagramme: 
\[\xymatrix@=0.5cm{
  & \eD(\cA) \\
  \eD^+(\cA) \ar[ur] 
    & & \eD^-(\cA) \ar[ul] \\
  & \eD^b(\cA) \ar[ul] \ar[ur]
}\]
étant injectifs sur les objets, réalisent les catégories $\eD^b(\cA)$, 
$\eD^+(\cA)$, $\eD^-(\cA)$ comme sous-catégories pleines de $\eD(\cA)$. 
\end{proposition}

La preuve de cette proposition se trouve au théorème \ref{VIII:2-4-2}. 





\begin{definition}\label{VIII:3-1-3}
La catégorie $\eD(\cA)$ sera appelée la \emph{catégorie dérivée} de la 
catégorie $\cA$. Les objets de $\eD(\cA)$ isomorphes aux objets de $\eD^b(\cA)$ 
seront appelés les objets \emph{bornés} de $\eD(\cA)$. Les objets de 
$\eD(\cA)$ isomorphes aux objets de $\eD^+(\cA)$ seront appelés les objets 
\emph{limités à gauche}, les objets de $\eD(\cA)$ isomorphes aux objets de 
$\eD^-(\cA)$, seront appelés les objets \emph{limités à droite}. 
\end{definition}

Nous désignerons par $D$ le foncteur canonique: 
\[
  D : \eC(\cA) \to \eD(\cA) \text{.}
\]
La ``restriction'' de $D$ à la sous-catégorie pleine $\cA$ de $\eC(\cA)$ 
sera encore noté $D$. Le foncteur $D$ restrient à $\cA$ est pleinement 
fidèle. Le foncteur $D$ se factorise d'une manière unique par $\eK(\cA)$. 
Soient $X$ et $Y$ deux objets de $\cA$. 

On vérifie sans difficulté que pour tout $m\geqslant 0$, les groupes: 
\[
  \hom_{\eD(\cA)}(D(X),T^{-m} D(Y)) \qquad \text{($T$ est le foncteur translation)} 
\]
sont nuls. La dimension cohomologique de $\cA$ sera le plus petit des entiers 
$n$ tels que pour tout $m>n$ on ait: 
\[
  \hom_{\eD(\cA)}(D(X),T^m D(Y)) = 0 \qquad\text{pout tout couple $X,Y$ d'objets de $\cA$.} 
\]
S'il n'y a pas de tels entiers, on dira que la dimension cohomologique de $\cA$ 
est infinie. 

Désignons par $\eI^+(\cA)$ (resp. $\eI(\cA)$) la sous-catégorie triangulée 
pleine de $\eK^+(\cA)$ (resp. de $\eK(\cA)$) définie par les complexes dont 
les objets sont injectifs en tout degré. 





\begin{proposition}\label{VIII:3-1-4}
Supposons que la catégorie $\cA$ possède suffisamment d'injectifs. Le 
foncteur naturel: 
\[
  Q^+:\eK^+(\cA) \to \eD^+(\cA) 
\]
induit une équivalence de catégorie: 
\[
  \eI^+(\cA) \to \eD^+(\cA)  \text{.}
\]
Le foncteur quasi-inverse est un adjoint à droite de $Q^+$. 
\end{proposition}

Si de plus $\cA$ est de dimension cohomologique finie, l'assertion 
précédente est encore vraie lorsqu'on y supprime les exposants $+$. 

On a énoncé analogue concernant les projectifs. 

La démonstration de la proposition \ref{VIII:3-1-4} s'appuie sur 
\ref{VIII:2-6-4}. 

La structure triangulée de $\eD(\cA)$, est précisée par la proposition 
suivante. 

Soit $\eE(\cA)$ la catégorie des suites exactes à trois termes de 
$\eC(\cA)$. La catégorie $\mathsf{SSS}(\cA)$ des suites exactes 
semi-scindées (\S\ref{VIII:1-2-4}) est une sous-catégorie pleine de 
$\eE(\cA)$. On a défini (Loc. cit.) un foncteur: 
\[
  \rho:\mathsf{SSS}(\cA) \to \mathsf{TrK}(\cA) 
\]
où $\mathsf{TrK}(\cA)$ désigne la catégorie des triangles distingués 
de $\eK(\cA)$. On en déduit un foncteur 
\[
  \sigma : \mathsf{SSS}(\cA) \to \mathsf{TrD}(\cA) 
\]
où $\mathsf{TrD}(\cA)$ désigne la catégorie des triangles distingués 
de $\eD(\cA)$. 





\begin{proposition}\label{VIII:3-1-5}
Il existe un et un seul foncteur: 
\[
  \Pi:\eE(\cA) \to \mathsf{TrD}(\cA) 
\]
de la forme ($\deg \delta(S)=1$): 
\[
(S=0 \to X^\bullet \to Y^\bullet \to Z^\bullet \to 0) 
\mapsto 
\xymatrix{
  & D(Z^\bullet) \ar[dl]_-{\delta(S)} \\
  D(X^\bullet) \ar[rr]^-{D(u)} 
    & & D(Y^\bullet) \ar[ul]_-{D(v)} 
}
\]
dont la restriction à $\mathsf{SSS}(\cA)$ soit $\sigma$. \emph{Ce foncteur 
est essentiellement surjectif.}
\end{proposition}





\begin{proposition}\label{VIII:3-1-6}
Le foncteur cohomologique canonique: 
\[
  \h^0:\eK(\cA) \to \cA 
\]
se factorise d'une manière unique par un foncteur que nous noterns encore 
$\h^0$ 
\[
  \h^0:\eD(\cA) \to \cA \text{.}
\]
Sur $\eD(\cA)$ le foncteur $\h^0$ et ses translatés $\h^i$ forment un 
système conservatif, i.e. un morphisme $f:X^\bullet \to Y^\bullet$ et un 
isomorphisme si et seulement si pour tout entier $i$ 
\[
  \h^i(f):\h^i(X) \to \h^i(Y) 
\]
est un isomorphisme. 
\end{proposition}





\begin{definition}\label{VIII:3-1-7}
Un morphisme $f:X^\bullet \to Y^\bullet$ de $\eC(\cA)$ (resp. de $\eK(\cA)$) 
est appelé un \emph{quasi-isomorphisme} lorsque pour tout entier $i$ 
\[
  \h^i(f):\h^i(X) \to \h^i(Y) 
\]
est un isomorphisme. 
\end{definition}

D'après la proposition précédente les quasi-isomorphismes sont les 
morphismes qui deviennent des isomorphismes dans $\eD(\cA)$. 





\begin{definition}\label{VIII:3-1-8}
Soient $\sigma$ un ensemble d'objets de $\cA$, $X^\bullet$ un objet de 
$\eC(\cA)$ (resp. $\eK(\cA)$). Une re\'solution droite (reps. gauche) de type 
$\sigma$ de $X^\bullet$, est un quasi-isomorphisme $X^\bullet \to V^\bullet$ 
(resp. $V^\bullet \to X^\bullet$) où $V^\bullet$ est un complexe dont tous 
les objets appartiennent à $\sigma$. On dira \emph{résolution injective} 
(resp. projective) au lieu de résolution droite (resp. gauche) de type 
injectif (resp. projectif). 
\end{definition}

De manière générale, on se permettra de supprimer les mots ``droite'' ou 
``gauche'' lorsqu'aucune ambiguité n'en résultera. 





\begin{proposition}\label{VIII:3-1-9}
\leavevmode
\begin{enumerate}
  \item Supposons que tout objet de $\cA$ s'injecte dans un objet de $\sigma$ 
    (resp. soit quotient d'un objet de $\sigma$). Alors tout objet de 
    $\eC^+(\cA)$ (resp. de $\eC^-(\cA)$) admet une résolution droite 
    (resp. gauche) de type $\sigma$ dans $\eC^+(\cA)$ (resp. dans 
    $\eC^-(\cA)$). 
  \item Supposons de plus qu'il existe un entier $n$ tel que pour toute suite 
    exacte: 
    \begin{align*}
      Y^0 &\to Y^1 \to \cdots \to Y^{n-1} \to Y^n \to 0 \\
      \text{(resp. } 0 &\to Y^n \to Y^{n-1} \to \cdots \to Y^1 \to Y^0 \text{)} 
    \end{align*}
    d'objets de $\cA$ ou pout tout $i$, $0\leqslant i\leqslant n-1$, $Y^i$ est 
    un objet de $\sigma$, l'objet $Y^n$ soit un objet de $\sigma$. Alors tout 
    objet de $\eC(\cA)$ admet une résolution droite (resp. gauche) de type 
    $\sigma$. 
\end{enumerate}
\end{proposition}










\subsection{Etude des $\ext$}\label{VIII:3-2}

Soient $X^\bullet$ et $Y^\bullet$ deux objets de $\eC(\cA)$ (resp. $\eK(\cA)$) 
et $D:\eC(\cA) \to \eD(\cA)$ (resp. $Q:\eK(\cA) \to \eD(\cA)$) le foncteur 
canonique. 





\begin{definition}\label{VIII:3-2-1}
On appelle \emph{$i$-ème hyper-ext} et on note: 
\[
  (X^\bullet,Y^\bullet)\mapsto \ext^i(X^\bullet,Y^\bullet) 
\]
le foncteur: $\hom_{\eD(\cA)}(D(X^\bullet),T^i D(Y^\bullet))$ (resp. 
$\hom_{\eD(\cA)}(Q(X^\bullet),T^i Q(Y^\bullet))$). 
\end{definition}





\begin{proposition}\label{VIII:3-2-2}
Soit $0 \to X^\bullet \to Y^\bullet \to Z^\bullet \to 0$ une suite exacte de 
$\eC(\cA)$. Soit $V^\bullet$ un objet de $\eC(\cA)$. On a les suites exactes 
illimitées: 
\[\xymatrix@=0.5cm{
  \cdots \ar[r] 
    & \ext^i(V^\bullet,X^\bullet) \ar[r] 
    & \ext^i(V^\bullet,Y^\bullet) \ar[r] 
    & \ext^i(V^\bullet,Z^\bullet) \ar[r]^-\delta 
    & \ext^{i+1}(V^\bullet,X^\bullet) \ar[r] 
    & \cdots \\
  \cdots \ar[r] 
    & \ext^i(Z^\bullet,V^\bullet) \ar[r] 
    & \ext^i(Y^\bullet,V^\bullet) \ar[r] 
    & \ext^i(X^\bullet,V^\bullet) \ar[r]^-\delta 
    & \ext^{i+1}(Z^\bullet,V^\bullet) \ar[r] 
    & \cdots
}\]
\end{proposition}

Soient $X^\bullet$ et $Y^\bullet$ deux objets de $\eK(\cA)$. On désigne par 
$\mathsf{Qis}/X^\bullet$ (resp. $\mathsf{Qis}^+/X^\bullet$, 
$\mathsf{Qis}^-/X^\bullet$, $\mathsf{Qis}^b/X^\bullet$) la catégorie des 
quasi-isomorphismes de but $X^\bullet$ et de source dans $\eK(\cA)$ (resp. 
$\eK^+(\cA)$, $\eK^-(\cA)$, $\eK^b(\cA)$). De même on définit les 
catégories $Y^\bullet/\mathsf{Qis}$, $Y^\bullet/\mathsf{Qis}^+$,\ldots 
(quasi-isomorphismes de source $Y^\bullet)$. 

On se propose de résumer dans la proposition suivante les différentes 
définitions équivalentes des $\ext^i$. 





\begin{proposition}\label{VIII:3-2-3}
1. Il existe des isomorphismes de foncteurs: 
\[\xymatrix{
  \ext^0(X^\bullet,Y^\bullet) = \hom_{\eD(\cA)}(Q(X^\bullet),Q(Y^\bullet)) \ar[r]^-\sim 
    & \varinjlim_{(\mathsf{Qis}/X^\bullet)^\circ} \hom_{\eK(\cA)}(-,Y^\bullet) \ar[r]^-\sim 
    & \\
  \varinjlim_{Y^\bullet/\mathsf{Qis}} \hom_{\eK(\cA)}(X^\bullet,-) \ar[r]^-\sim 
    & \varinjlim_{(\mathsf{Qis}/X^\bullet)^\circ \times Y^\bullet/\mathsf{Qis}} \hom_{\eK(\cA)}(-,-) \text{.}
}\]
Si $X^\bullet$ est un objet de $\eK^-(\cA)$: 
\[
  \ext^0(X^\bullet,Y^\bullet) \iso \varinjlim_{(\mathsf{Qis}^-/X^\bullet)^\circ} \hom_{\eK(\cA)}(-,Y^\bullet) 
\]
Si $Y^\bullet$ est un objet de $\eK^+(\cA)$: 
\[
  \ext^0(X^\bullet,Y^\bullet) \iso \varinjlim_{Y^\bullet/\mathsf{Qis}} \hom_{\eK(\cA)}(X^\bullet,-) 
\]
Si $X^\bullet$ est un objet de $\eK^-(\cA)$ et si $Y^\bullet$ est un 
objet de $\eK^b(\cA)$: 
\[
  \ext^0(X^\bullet,Y^\bullet) \iso \varinjlim_{Y^\bullet/\mathsf{Qis}^b} \hom_{\eK(\cA)}(X^\bullet,-) 
\]
Si $X^\bullet$ est un objet de $\eK^b(\cA)$ et $Y^\bullet$ un objet de 
$\eK^+(\cA)$: 
\[
  \ext^0(X^\bullet,Y^\bullet) \iso \varinjlim_{(\mathsf{Qis}^b/X^\bullet)^\circ} \hom_{\eK(\cA)}(-,Y^\bullet) 
\]
Si la catégorie $\cA$ possède suffisamment d'injectif et si 
$Y^\bullet$ est un objet de $\eK^+(\cA)$, $Y^\bullet$ admet une 
résolution injective et une seule (à isomorphisme près dans 
$\eK^+(\cA)$): 
\[
  Y^\bullet \to I(Y^\bullet) 
\]
et on a: 
\[
  \ext^0(X^\bullet,Y^\bullet) \iso \hom_{\eK(\cA)}(X^\bullet,I(Y^\bullet)) \text{.}
\]
On a de même un énoncé analogue pour les projectifs. 
Si la catégorie $\cA$ admet suffisamment d'injectifs et si $\cA$ est 
de \emph{dimension cohomologique finie}, tout objet $Y^\bullet$ de 
$\eK(\cA)$ admet une résolution injective et une seule: 
\[
  Y^\bullet \to I(Y^\bullet)
\]
et on a: 
\[
  \ext^0(X^\bullet,Y^\bullet) \iso \hom_{\eK(\cA)}(X^\bullet,I(Y^\bullet)) 
\]
Enoncé analogue pour les projectifs. 
\end{proposition}

\paragraph*{Remarque:}
En explicitant l'isomorphisme de la proposition \ref{VIII:3-2-3}(3), on 
retrouve la définition de Yoneda. 










\section{Les foncteurs dérivés}\label{VIII:4}










\subsection{Définition des foncteurs dérivés}\label{VIII:4-1}





\begin{definition}\label{VIII:4-1-1}
Soient $\cC$ et $\cC'$ deux catégories graduées (on désigne par $T$ le 
foncteur de translation de $\cC$ et de $\cC'$), $F$ et $G$ deux foncteurs 
gradués de $\cC$ dans $\cC'$. Un \emph{morphisme de foncteurs gradués} et 
un morphisme de foncteurs: 
\[
  u:F\to G 
\]
qui possède la propriété suivante: 

Pour tout objet $X$ de $\cC$ le diagramme suivant est commutatif: 
\[\xymatrix{
  F(T X) \ar[r]^-{u(T X)} 
    & G(T X) \\
  T F(X) \ar[r]^-{T u(X)} \ar[u]_-\wr 
    & T G(X) \ar[u]_-\wr 
}\]
\end{definition}

Soient $\cC$ et $\cC'$ deux catégories triangulées. On désigne par 
$\mathsf{Fex}(\cC,\cC')$ la catégorie des foncteurs exacts de $\cC$ dans 
$\cC'$, le morphismes entre deux foncteurs étant les morphismes de foncteurs 
gradués. 

Soient $\cA$ et $\cB$ deux catégories abéliennes de 
$\Phi:\eK^\ast(\cA)\to \eK^{\ast'}(\cB)$ un foncteur exact ($\ast$ et $\ast'$ 
désignent l'un des signes $+$, $-$, $b$, ou $v$ ``vide''). Le foncteur 
canonique $Q:\eK^\ast(\cA)\to \eD^\ast(\cA)$ nous donne, par composition, un 
foncteur 
\[\xymatrix{
  \mathsf{Fex}(\eD^\ast(\cA),\eD^{\ast'}(\cB)) \ar[r] 
    & \mathsf{Fex}(\eK^\ast(\cA),\eD^{\ast'}(\cB)) 
}\]
d'où (en désignant aussi par $Q'$ le foncteur canonique 
$\eK^{\ast'}(\cB) \to \eD^{\ast'}(\cB)$) un foncteur: 
$\varkappa$ (resp. $\varkappa'$): 
$\mathsf{Fex}(\eD^\ast(\cA),\eD^{\ast'}(\cB)) \to \mathsf{Ab}$: 
\begin{align*}
  \Psi  &\mapsto \hom(Q'\circ \Phi,\Psi\circ Q) \\
  (\text{resp. }\Psi &\mapsto \hom(\Psi\circ Q,Q'\circ \Phi)\text{ ).} 
\end{align*}





\begin{definition}\label{VIII:4-1-2}
On dira que $\Phi$ admet un \emph{functeur dérivé total} à droite (resp. 
à gauche) si le foncteur $\varkappa$ (resp. $\varkappa'$) est représentable. Un objet 
représentant le foncteur $\varkappa$ (resp. $\varkappa'$) ser appelé foncteur dérivé 
total à droite (resp. à gauche) de $\Phi$ et sera noté 
$\eR \Phi$ (resp. $\eL \Phi$).\footnote{Une définition plus maniable (et en 
practique équivalente) est donnée dans \cite[XVII 1.2]{sga4}.}
\end{definition}





\subsubsection{Notations}\label{VIII:4-1-3}

Soient $\cA$ et $\cB$ deux catégories abéliennes et $f:\cA\to \cB$ un 
foncteur additif. Le \emph{foncteur dérivé total à droite} du foncteur 
$\eK^+(f):\eK^+(\cA) \to \eK^+(\cB)$ sera noté $\eR^+ f$. On définit de 
même $\eR^- f$, $\eR f$, $\eL^+ f$, $\eL^- f$, $\eL f$. 

Soit de même $F:\eK^\ast(\cA) \to \cB$, un foncteur cohomologique. Le 
foncteur canonique de passage au quotient $Q:\eK^\ast(\cA) \to \eD^\ast(\cA)$ 
définit par composition un foncteur: 
\[\xymatrix{
  \mathsf{Foco}(\eD^\ast(\cA),\cB) \ar[r] 
    & \mathsf{Foco}(\eK^\ast(\cA),\cB) 
}\]
($\mathsf{Foco}(-,-)$ désigne la catégorie des foncteurs cohomologiques) 
d'où un foncteur $\varkappa$ (resp. $\varkappa'$) 
$\mathsf{Foco}(\eD^\ast(\cA),\cB) \to \mathsf{Ab}$, $G\mapsto \hom(F,G\circ Q)$ 
(resp. $G\mapsto \hom(G\circ Q,F)$). 





\begin{definition}\label{VIII:4-1-4}
On dira que le foncteur $F$ admet un \emph{foncteur dérivé cohomologique} 
à droite (resp. à gauche) si le foncteur $\varkappa$ (resp. $\varkappa'$) est 
représentable. Un objet représentant le foncteur $\varkappa$ (resp. $\varkappa'$) sera 
appelé foncteur dérivé cohomologique à droite (resp. à gauche) et 
sera noté $\eR F$ (resp. $\eL F$). 
\end{definition}





\subsubsection{Notations}\label{VIII:4-1-5}

Soient $\cA$ et $\cB$ deux catégories abéliennes et $f:\cA\to \cB$ un 
foncteur additif. Le foncteur dérivé cohomologique à droite du foncteur: 
\[
  \h^0{}\circ \eK^+ f:\eK^+(\cA) \to \cB 
\]
sera noté $\eR^+ f$. Lorsqu'aucune confusion n'en résultera le foncteur 
$\eR^+ f\circ T^i$ ($T$ est le foncteur de translation) sera noté: 
$\eR^i f$. Le foncteur dérivé cohomologique à droite du foncteur 
$\h{}\circ \eK f$ sera noté $\eR f$. Lorsqu'aucune confusion n'en résultera 
le foncteur $\eR f\circ T^i$ sera noté $\eR^i f$. 

On définit de même les foncteurs $\eL^- f$, $\eL f$, $\eL^i f$. 





\subsubsection{Remarques}\label{VIII:4-1-6}

Ces définitions contiennent, ainsi que nous le verrons dans le numéro 
suivant, la définition classique des foncteurs dérivés lorsque les 
catégories ont suffisamment d'injectifs ou de projectifs et que les complexes 
sont convenablement limités. Elles sont cependant plus générales. Mais 
sans autres hypothèses elles sont peu maniables. Par exemple, soit 
$f:\cA\to \cB$ un foncteur additif entre deux catégories abéliennes. 
Supposons que les foncteurs $\eR^+ f$ et $\eR f$ existent. Je ne sais pas 
d\'montrer que $\eR f$ induit sur $\eD^+(\cA)$ le foncteur $\eR^+ f$. Supposons 
de plus que le foncteur $\eR^+ f$ (dérivé cohomologique) existe. Je ne sais 
pas démontrer que le foncteur $\h^0{}\circ \eR^+ f$ est isomorphe au foncteur 
$\eR^+ f$. Cependant dans la pratique ces propriétés seront vérifées. 










\subsection{Existence des foncteurs dérivés}\label{VIII:4-2}





\begin{proposition}\label{VIII:4-2-1}
Les hypothèses sont celles de la définition \ref{VIII:4-1-4}. Supposons 
en outre que $\cA$ soit un $\sU$-catégorie, où $\sU$ est un universe tel 
que l'ensemble $\dZ$ des entiers rationnel soit un élément de $\sU$, telle 
que l'ensemble des objets de $\cA$ soit un élément de $\sU$. Supposons 
de plus que la catégorie $\cB$ soit la catégorie des $\sU$-groupes 
abéliens ou plus générelement la catégorie des faisceaux de 
$\sU$-groupes abéliens sur un $\sU$-site quelconque. (Un $\sU$-site est une 
$\sU$-catégorie, dont l'ensemble des onjets est un élément de $\sU$, 
munie d'une topologie.) Le foncteur $F$ admet un foncteur dérivé 
cohomologique à droite. Ce foncteur se calcule de la manière suivante: 

Soient $X$ un objet de $\eD^\ast(\cA)$, $\underline X$ l'objet de 
$\eK^\ast(\cA)$ au-dessus de $X$. On a alors un isomorphisme fonctoriel: 
\[
  \eR F(X) = \varinjlim_{\underline X/\mathsf{Qis}^\ast} F(-) \text{.}
\]
\end{proposition}

En particulier soit $f:\cA\to \cB$ un foncteur additif. Les foncteurs 
$\eR f$ et $\eR^+ f$ existent et se calculent de la manière indiquée 
ci-dessus. Le foncteur $\eR f$ induit sur $\eD^+(\cA)$ le foncteur 
$\eR^+ f$. Comme aucune confusion ne peut alors en résulter, nous 
emploierons la notation $\eR^i f$. 





\subsubsection{Remarques}\label{VIII:4-2-2}

On peut exprimer la proposition \ref{VIII:4-2-1} sous une forme plus 
générale dans le cadre des catégories triangulées. Les hypothèses 
à faire sur la catégorie $\cB$, pour assurer la validité de la 
proposition \ref{VIII:4-2-1} sont des hypothèses d'existence et d'exactitude 
des limites inductives suivant les $\sU$-catégories pseudo-filtrantes, dont 
les ensembles d'objet sont éléments de $\sU$. La proposition 
\ref{VIII:4-2-1} introduit une dissymétrie entre les foncteurs dérivés 
droits et les foncteurs dérivés gauches tout au moins dans la pratique. 





\begin{theorem}\label{VIII:4-2-3} % labelled 4.2.2 in original
Les données sont celles de la définition \ref{VIII:4-1-2}. On se donne de 
plus une sous-catégorie triangulée pleine de $\eK^\ast(\cA)$: $\cC$, et on 
d\'signe par $\cN$ la sous-catégorie triangulée pleine des objets de $\cC$ 
qui sont acycliques. Supposons que 
\begin{enumerate}[(1)]
  \item Tout objet de $\cN$ soit transformé par $\Phi$ en objet acyclique 
    de $\eK^{\ast'}(\cB)$. 
  \item Tout objet $X$ de $\eK^\ast(\cA)$ soit source (resp. but) d'un 
    quasi-isomorphisme dont le but (resp. la source) soit un objet de $\cC$. 
\end{enumerate}
\begin{enumerate}[a)]
  \item Alors $\Phi$ admet un foncteur dérivé total à droite (resp. 
    à gauche). 
  \item Le fonteur $\eR \Phi$ (resp. $\eL \Phi$) s'obtient de la manière 
    suivante: 
    
    La restriction du foncteur 
    $Q'\circ \Phi:\eK^\ast(\cA) \to \eD^{\ast'}(\cB)$ à la catégorie $\cC$ 
    s'annule sur les objets de $\cN$, donc se factorise par $\cC/\cN$. On 
    obtient ainsi un foncteur $\Phi':\cC/\cN\to \eD^{\ast'}(\cB)$. Mais le 
    foncteur naturel: $\cC/\cN\to \eD^\ast(\cA)$ est une équivalence de 
    catégories (th\'or\`me \ref{VIII:2-4-2}). D'où en composant avec le 
    foncteur quasi-inverse le foncteur $\eR\Phi$ (resp. $\eL \Phi$). 
  \item Soit $X$ un objet de $\eK^\ast(\cA)$. Soient $Y$ un objet de $\cC$ et 
    $X\to Y$ (resp. $Y\to X$) un quasi-isomorphisme. L'objet 
    $\eR\Phi\circ Q(X)$ (resp. $\eL \Phi\circ Q(X)$) est isomorphe à 
    $Q\circ \Phi(Y)$. 
  \item Le foncteur $X\mapsto \eR\Phi\circ Q(X)$ (resp. 
    $X\mapsto \eL \Phi\circ Q(X)$) est isomorphe au foncteur: 
    $X\mapsto \varinjlim_{X/\mathsf{Qis}^\ast} Q\circ\Phi(-)$ (resp. 
    $X\mapsto \varinjlim_{\mathsf{Qis}^\ast/X} Q\circ \Phi(-)$). 
  \item Le foncteur $\h^0{}\circ \Phi:\eK^\ast(\cA)\to \cB$, admet un 
    foncteur dérivé cohomologique à droite (resp. à gauche) qui n'est 
    autre que le foncteur $\h^0{}\circ\eR\Phi$ (resp. $\h^0{}\circ \eL\Phi$). 
\end{enumerate}
\end{theorem}





\begin{corollary}\label{VIII:4-2-3-1} % originally corollary 1
Supposons que $\ast=+$ (resp. $\ast=-$) dans les hypothèses du théorème 
précédent, et que $\cA$ possède suffisamment d'injectifs (resp. de 
projectifs). Alors le théorème \ref{VIII:4-2-3} s'applique en prenant pour 
$\cC$ la catégorie des complexes injectifs (resp. projectifs). Supposons de 
plus que $\cA$ soit de dimension cohomologique finie, alors le théorème 
s'applique, sans restriction sur $\ast$, en prenant sur $\cC$ la catégorie 
des complexes injectifs (resp. projectifs). 
\end{corollary}





\begin{definition}\label{VIII:4-2-5} % originally 2.4
Soient $\cA$ et $\cB$ deux catégories abéliennes et $f:\cA\to \cB$ un 
foncteur additif. 

On dira que $f$ est de \emph{dimension cohomologique finie} à droite (resp. 
à gauche) si $\eR^+ f$ (resp. $\eL^- f$) existe et s'il existe un entier 
$m\geqslant 0$ tel que pour tout entier $n>m$, $\eR^n f$ (resp. 
$\eL^{-n} f$) soit nul sur les objets provenant de $\cA$ par le foncteur $D$. 
La \emph{dimension cohomologique} à droite (resp. à gauche) de $f$ est 
alors le plus petit des entiers $m$ possédant la propriété ci-dessus. Si 
$\eR^+ f$ (resp. $\eL^- f$) existe et si $f$ n'est pas de dimension 
cohomologique finie, on dira que $f$ est de dimension cohomologique infinie. 

Un objet $X$ de $\cA$ sera dit \emph{$f$-acyclique} à droite (resp. à 
gauche) si pour tout entier $n>0$, $\eR^n f(D(X)) = 0$ (resp. 
$\eL^{-n} f(D(X))=0$). 
\end{definition}





\begin{corollary}\label{VIII:4-2-3-2} % originally corollary 2
Les données sont celles de la définition \ref{VIII:4-2-5}. Supposons qu'il 
existe un ensemble $\sigma$ d'objets de $\cA$ stable par somme directe 
possédant les propriétés suivantes: 
\begin{enumerate}
  \item Tout complexe acyclique $\in \operatorname{Ob}{\eK^+(\cA)}$ (resp. 
    $\in\operatorname{Ob}{\eK^-(\cA)}$) d'objets de $\sigma$, est transformé 
    par $f$ en complexe acyclique. 
  \item Tout objet $X$ de $\cA$ s'injecte dans (resp. est quotient de) un objet 
    de $\sigma$. Les hypothèses du théorème \ref{VIII:2-4-3} sont 
    vérifiées en prenant $\ast=+$ (resp. $\ast=-$) et en prenant pour 
    catégorie $\cC$, la catégorie des complexes dont les objets et tout 
    degré sont des éléments de $\sigma$. Les éléments de $\sigma$ 
    sont des objets $f$-acycliques à droite (resp. à gauche). 
\end{enumerate}

Si de plus $f$ est de dimension cohomologique finie à droite (resp. à 
gauche), l'ensemble des objets $f$-acycliques à droite (resp. à gauche) 
[pssède en plus des propriétés (1) et (2) ci-dessus, la propriété 
(2) de proposition \ref{VIII:3-1-9}. Le Théorème \ref{VIII:4-2-3} 
s'applique alors au foncteur $\eK f$ en prenant pour catégorie $\cC$ la 
catégorie des complexes dont les objets sont $f$-acycliques à droite 
(resp. à gauche) en tout degré. Le foncteur $\eR f$ (resp. $\eL f$) existe 
et induit sur $\eD^+(\cA)$ le foncteur $\eR^+ f$ (resp. $\eL^+ f$) et sur 
$\eD^-(\cA)$ le foncteur $\eR^- f$ (resp. $\eL^- f$). 
\end{corollary}










\subsection{Produit. Composition}\label{VIII:4-3}

Soient $\cA$ une catégorie abélienne, $F:\eD(\cA) \to \mathsf{Ab}$ un 
foncteur cohomologique. Soient $X^\bullet$ et $Y^\bullet$ deux objets de 
$\eC(\cA)$. Le foncteur $F$ définit une application: 
\[
  \ext^0(X^\bullet,Y^\bullet) \to \hom_{\mathsf{Ab}}(F(D(X^\bullet)),F(D(Y^\bullet)))
\]
d'où un accouplement: 
$F(D(X^\bullet))\times \ext^0(X^\bullet,Y^\bullet) \to F(D(Y^\bullet))$. 

Cet accouplement donne, en appliquant des translations, des accouplements: 
\[
  F^i(D(X^\bullet))\times \ext^j(X^\bullet,Y^\bullet) \to F^{i+j}(D(Y^\bullet)) \text{.}
\]
Ces accouplements, appliqués aux foncteurs dérivés cohomologiques, ne 
sont autres, par définition, que les produits de Yoneda. Les propriétés 
de ces produits se déduisent immédiatement de cette définition. Nous ne 
les développerons pas ici. 

Soient $\cA$, $\cA'$, $\cA''$ trois catégories abéliennes et 
$f:\cA\to \cA'$, $g:\cA'\to \cA''$ deux foncteurs additifs. Supposons que les 
foncteurs $\eR^+ f$ et $\eR^+ g$ existent. Je ne sais pas en déduire que le 
foncteur $\eR^+ g\circ f$ existe et même si ce dernier foncteur existe, il 
n'est probablement pas vraie en général que l'on ait la formule: 
\[\xymatrix{
  \eR^+ g\circ f\ar[r]^-\sim 
    & \eR^+ g \circ \eR^+ f 
}\]
On a cependant la proposition suivante: 





\begin{proposition}\label{VIII:4-3-1}
Supposons que la catégorie $\cA'$ possède suffisamment d'objets 
$g$-acycliques à droite et que la catégorie $\cA$ possède suffisamment 
d'objets $f$-acycliques à droite transformés par $f$ en objets 
$g$-acycliques à droite. Alors le foncteur $\eR^+ g\circ f$ existe et on a un 
isomorphisme: 
\[\xymatrix{
  \eR^+ g\circ f \ar[r]^-\sim 
    & \eR^+ g \circ \eR^+ f \text{.}
}\]
\end{proposition}

Enoncé analogues pour les foncteurs dérivés gauches. 















\section{Exemples}\label{VIII:6} % originally chapter 2, section 3










\subsection{Le foncteur \texorpdfstring{$\mathsf{Rhom}^\bullet$}{Hom*}}\label{VIII:6-1}

Soit $\cA$ une catégorie abélienne, possédant suffisamment d'injectifs. 

Le foncteur: $Y\mapsto \hom^\bullet(X^\bullet,Y^\bullet)$ (\ref{VIII:1-3}) 
où $X^\bullet$ est un objet de $\eK(\cA)$ et $Y^\bullet$ un objet de 
$\eK^+(\cA)$, admet un foncteur dérivé total à droite 
(Corollaire \ref{VIII:4-2-3-1}) qui sera noté $\rhom^\bullet(X^\bullet,-)$.  
Soit $Y$ un objet de $\eD^+(\cA)$ le foncteur 
\begin{align*}
  \eK(\cA) &\to \eD(\mathsf{Ab}) \\
  X^\bullet &\mapsto \rhom^\bullet(X^\bullet,Y) 
\end{align*}
est exact et s'annule sur les complexes acycliques, d'où un bi-foncteur 
exact 
\[\xymatrix{
  \eD(\cA)^\circ \times \eD^+(\cA) \ar[r] 
    & \eD(\mathsf{Ab}) 
}\]
que nous désignerons par $\mathsf{Rhom}^\bullet$. 

Lorsque $\cA$ est de dimension cohomologique finie, le foncteur 
$\mathsf{Rhom}^\bullet$ se prolonge à $\eD(\cA)^\circ\times \eD(\cA)$ 
(loc. cit.). 

Si en outre $\cA$ possède suffisamment de projectifs on peut dériver le 
foncteur $\hom$ par rapport au premier argument. Les deux définitions 
co\"incident dans leur domaine commun de validité. 










\subsection{\texorpdfstring{$\Tor$}{Tor} de faisceaux}\label{VIII:6-2}

Soient $E$ un site annelé (en particulier un espace topologique annelé) 
$\sA$ le faisceau d'anneaux et $\mathsf{Mod}(E)$ la catégorie des 
$\sA$-modules. Un objet $X$ de $\mathsf{Mod}(E)$ est dit 
\emph{$\sA$-plat} si pout toute suite exacte: 
\[\xymatrix{
  0 \ar[r] 
    & Y' \ar[r] 
    & Y \ar[r] 
    & Y'' \ar[r] 
    & 0 
}\]
la suite: $0 \to Y'\otimes_\sA X \to Y\otimes_\sA X \to Y''\otimes_\sA X\to 0$ 
est exacte. Il existe suffisamment d'objets $\sA$-plats: les sommes directes 
d'objets $\sA_U$, où $U$ est un objet de $E$ et $\sA_U$ le faisceau 
$\sA$ ``prolongué par zéro en dehors de $U$.'' Le théorème 
\ref{VIII:4-2-3} s'applique donc et on peut définir le foncteur 
\begin{align*}
  \eD^-(E,\sA) \times \eD^-(E,\sA) &\to \eD^-(E,\sA) \\
  (X,Y) &\mapsto X\lotimes Y 
\end{align*}
produit tensoriel total, dérivé gauche du foncteur produit tensoriel. 
On définit ainsi en passant à la cohomologie les $\Tor_i^\sA(X,Y)$: 
hyper-tor locaux. 

La catégorie $\mathsf{Mod}(E)$ ayant suffisamment d'injectifs, on peut 
dériver le foncteur: $\hom^\bullet(X,Y)$: complexe des homomorphismes locaux,  
d'où $\mathsf{Rhom}(X,Y)$ ($X\in \eD$, $Y\in \eD^+$). 

Soient $X$ et $Y$ deux objets de $\eD^-(E,\sA)$, $Z$ un objet de 
$\eD^+(E,\sA)$. Il existe un isomorphisme fonctoriel: 
\[\xymatrix{
  \hom_{\eD(E,\sA)}(X\lotimes Y,Z) \ar[r]^-\sim 
    &\hom_{\eD(E,\sA)}(X,\mathsf{Rhom}(Y,Z)) \text{.}
}\]










\subsection{Foncteur dérivé gauche de l'image réciproque}\label{VIII:6-3}

Soit $f:E\to E'$ une application d'espaces topologiques annelés. Le 
foncteur image réciproque: 
\[
  f^\ast:\mathsf{Mod}(E') \to \mathsf{Mod}(E) 
\]
n'est pas nécessairement exact. Mais on sait définir des objets 
$f^\ast$-plats et il existe suffisamment de tels objets. On sait donc 
définir le foncteur $\eL^- f^\ast$. On a de même une formule de dualité: 

Soient $X$ un objet de $\eD^-(E,\sA)$ et $Y$ un objet de 
$\eD^+(E',\sA')$; il existe un isomorphisme fonctoriel: 
\[\xymatrix{
  \hom_{\eD(E',\sA')}(\eL^- f^\ast(X),Y) \ar[r]^-\sim 
    & \hom_{\eD(E,\sA)}(X,\eR^+ f_\ast(Y)) \text{.}
}\]
















